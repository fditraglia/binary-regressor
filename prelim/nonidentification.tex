%!TEX root = ./main.tex

\subsection{Lack of Identification From Conditional Means}
We have seen that \cite{Mahajan}'s approach based on a binary instrument cannot identify $\beta$ when the treatment is endogenous: Assumption \ref{assump:Eq11} in fact implies that the instrument is \emph{irrelevant}. 
We now show that, regardless of how many values the instrument takes on, conditional mean information is insufficient for identification.

To begin, consider the model in Equation \ref{eq:linear} without any restrictions on the $m^*_{tk}$, that is \emph{without} imposing the IV restriction given in Equation \ref{eq:IV}.
In this fully general case, the $2K + 3$ unknown parameters are $\beta, \alpha_0, \alpha_1$ and the conditional means of $u$, namely $m^*_{tk}$.
In contrast, there are only $2K$ available moment conditions.
\begin{lem}
  Suppose that $\mathbb{E}\left[ \varepsilon|T^*,T,z \right]= \mathbb{E}\left[ \varepsilon|T^*,z \right]$. Then, under Assumption \ref{assump:A2} (ii)--(iv),
  \label{lem:eq1}
\begin{align}
  \label{eq:MC0}
  \hat{y}_{0k} &=\frac{\alpha_1(p_k - \alpha_0)(\beta + m_{1k}^*) + (1 - \alpha_0)(1 - p _k -  \alpha_1)m_{0k}^*}{1 - \alpha_0 - \alpha_1} \\[1.5ex]
  \label{eq:MC1}
  \hat{y}_{1k} &= \frac{(1-\alpha_1)(p_k - \alpha_0)(\beta+m_{1k}^*)+  \alpha_0(1-p_k - \alpha_1)m_{0k}^*}{1-\alpha_0 - \alpha_1}
\end{align}
where $\hat{y}_{0k} = (1-p_k)\bar{y}_{0k}$ and $\hat{y}_{1k}= p_k \bar{y}_{1k}$.
\end{lem}
Notice that the observable ``weighted'' cell mean $\hat{y}_{tk}$ defined in the preceding lemma depends on both $m^*_{tk}$ \emph{and} $m^*_{1-t,k}$ since the cell in which $T=t$ from Table \ref{tab:observables} is in fact a mixture of both the cells $T^*=0$ and $T^*=1$ from Table \ref{tab:unobservables}, for a particular column $k$.
Clearly we have fewer equations than unknowns.
What additional restrictions could we consider imposing on the system? 
In a very interesting paper, \cite{Lewbel} proposes using a three-valued ``instrument'' that does \emph{not} satisfy the exclusion restriction.
By assuming instead that there is no \emph{interaction} between the instrument and the treatment, he is able to prove identification of the treatment effect.
Using our notation it is easy to see why \cite{Lewbel} requires a three-valued instrument. 
His moment conditions are equivalent to Equations \ref{eq:MC0} and \ref{eq:MC1} with the additional restriction that $m^*_{0k} = m^*_{1k}$ for all $k= 1, \dots, K$.
This leaves the number of equations unchanged at $2K$, but reduces the number of unknowns to $K+3$.
The smallest $K$ for which $K+3$ is at least as large as $2K$ is 3.\footnote{The context considered by \cite{Lewbel} is slightly different from the one we pursue here, in that his ``instrument'' is more like a covariate: it is allowed to have a direct effect on the outcome of interest.  For this reason, \cite{Lewbel} cannot use the exogeneity of the treatment to obtain identification based on a two-valued instrument.}

Unlike \cite{Lewbel} we, along with \cite{Mahajan} and others, assume that $\mathbb{E}[\varepsilon|z]=0$ so that Equation \ref{eq:IV} holds. 
\begin{cor}
  \label{cor:eq2}
  Suppose that $\mathbb{E}\left[ \varepsilon|T^*,T,z \right]= \mathbb{E}\left[ \varepsilon|T^*,z \right]$. Then, under Assumption \ref{assump:A2},
\begin{align}
  \label{eq:MC0IV}
  \hat{y}_{0k} &=\alpha_1(p_k - \alpha_0)\left(\frac{\beta}{1 - \alpha_0 - \alpha_1}\right) + (1-\alpha_0)c - (p _k -  \alpha_0)m_{1k}^* \\[1.5ex]
  \label{eq:MC1IV}
  \hat{y}_{1k} &=(1-\alpha_1)(p_k - \alpha_0)\left(\frac{\beta}{1 - \alpha_0 - \alpha_1}\right) + \alpha_0 c + (p _k -  \alpha_0)m_{1k}^*
\end{align}
where $\hat{y}_{0k} = (1-p_k)\bar{y}_{0k}$ and $\hat{y}_{1k}= p_k \bar{y}_{1k}$.
\end{cor}
Equations \ref{eq:MC0IV} and \ref{eq:MC1IV} also make it clear why the IV estimator is inconsistent in the face of non-differential measurement error, and that this inconsistency does not depend on the endogeneity of the treatment, as noted by \cite{FL}.
Adding together Equations \ref{eq:MC0IV} and \ref{eq:MC1IV} yields
\begin{equation*}
  \hat{y}_{0k} + \hat{y}_{1k} = c + (p_k - \alpha_0)\left( \frac{\beta}{1 - \alpha_0 - \alpha_1} \right) 
\end{equation*}
completely eliminating the $m^*_{1k}$ from the system.
Taking the difference of the preceding expression evaluated at two different values of the instrument, $z_{k}$ and $z_{\ell}$, and rearranging
\begin{equation}
  \mathcal{W} = \frac{(\hat{y}_{0k} + \hat{y}_{1k}) - (\hat{y}_{0\ell} + \hat{y}_{1\ell})}{p_k - p_\ell} =  \frac{\beta}{1 - \alpha_0 - \alpha_1}
  \label{eq:wald}
\end{equation}
which is the well-known Wald IV estimator, since $\hat{y}_{0k} + \hat{y}_{1k} = \mathbb{E}[y|z = z_k]$.

Imposing $\mathbb{E}[\varepsilon|z]=0$ replaces the $K$ unknown parameters $\left\{ m^*_{0k}\right\}_{k=1}^K$ in Equations \ref{eq:MC0}--\ref{eq:MC1} with a single parameter $c$, leaving us with the same $2K$ equations but only $K+4$ unknowns.
When $K=2$ (a binary instrument) we have 4 equations and 6 unknowns.
So how can one identify $\beta$ in this case?
The literature has imposed additional assumptions which, using our notation, can once again be mapped into restrictions on the $m_{tk}^*$.
\cite{BBS}, \cite{KRS}, and \cite{Mahajan} make a \emph{joint} exogeneity assumption on $(T^*,z)$, namely $\mathbb{E}[\varepsilon|T^*,z]=0$.
Notice that this is strictly stronger than assuming that the instrument is valid and the treatment is exogenous.
In our notation, this joint exogeneity assumption is equivalent to imposing $m_{tk}^*=c$ for all $t,k$.
This reduces the parameter count to 4 regardless of the value of $K$.
Thus, when the instrument is binary, we have exactly as many equations as unknowns.
The arguments in \cite{BBS}, \cite{KRS}, and \cite{Mahajan} are all equivalent to solving Equations \ref{eq:MC0IV} and \ref{eq:MC1IV} for $\beta$ under the added restriction that $m^*_{1k}=c$, establishing identification for this case.
\cite{FL} use the same argument in a linear model with a potentially continuous instrument, but impose only the weaker conditions that the treatment is exogenous and the instrument is valid. 
Nevertheless, a crucial step in their derivation implicitly assumes the stronger joint exogeneity assumption used by \cite{BBS}, \cite{KRS} and \cite{Mahajan}.
Without this assumption, their proof does not in fact go through.
%Add a footnote showing the gap between the two sets of assumptions and how it relates to the continuous case, triple moment, etc.

If one wishes to allow for an endogenous treatment, the joint exogeneity assumption $m_{tk}^*=c$ is unusable and we have $2K$ equations in $K+4$ unknowns.
Based on the identification arguments described above, there would seem to be two possible avenues for identification of the treatment effect when a valid instrument is available.
One idea would be to impose alternative conditions on the $m^*_{tk}$ that are compatible with an endogenous treatment.
If $z$ is binary, two additional restrictions would suffice to equate the counts of moments and unknowns.
As we showed in Proposition \ref{pro:FirstStage}, however, this approach fails. 
Another idea, inspired by \cite{Lewbel}, would be to rely on an instrument that takes on more than two values.
Following this approach would suggest a 4-valued instrument, the smallest value of $K$ for which $2K = K+4$.
Unfortunately this approach fails as well, as we now show.
%\cite{FL} point out that an instrument can be used in place of a second measure of $T^*$ provided that $T^*$ is still exogenous.
%Essentially the same estimator as in \cite{BBS} and \cite{KRS} but more general since the instrument need not be binary: can in fact be continuous.
%But they make a mistake. 
%They assume $E[zu]=0$, $E[T^*u]=0$ and non-differential measurement error and claim that this is sufficient to consistently estimate $\beta$.
%However, this is incorrect: we need the additional assumption that $E[zT^*u]=0$ which is stronger.
%(They seem to think that this term only appears when you have an endogenous $T^*$.)
%While \cite{FL} are aware that there are some differences between two measures of $T^*$, as in \cite{BBS}, and an arbitrary instrument $z$, they seem to have missed one subtle point.
%The assumptions in \cite{BBS} in fact imply that $E[u|T^*,z]=0$.\footnote{This follows from Assumptions A1 and A2 combined with Equation 3.}
%From this it follows that $E[zT^*u]=E[zu]=E[T^*u]=0$. 
%However, if one takes the non-differential measurement assumption literally it is in fact sufficient in the case of two measures to assume only that $E[zu]=E[T^*u]=0$:
%\begin{align}
%  E[zT^*u] &= E[(T^*+w)T^*u] = E[(T^*)^2u] + E[wuT^*]  \\
%  &= E\left[ E\left( u|T^* \right)(T^*)^2 \right] + E\left[E\left( wu|T^* \right)T^*\right]\\
%  &= 0 + E\left[ E(w|T^*)E(u|T^*)T^* \right] = 0
%\end{align}
%using the fact that $E[u|T^*]=0$ and $w$ is independent of $u$ conditional on $T^*$.
%This argument does \emph{not} necessarily apply to an arbitrary instrument $z$: $E[zu]=E[T^*u]=0$ does not imply that $E[zT^*u]=0$.
%\todo[inline]{Put in our simple binary example.}
%While it might seem strange to assume in practice that $E[zu]=E[T^*u]=0$ are exogenous but not that $E[zT^*u]=0$ the point is merely that this is an additional assumption beyond the usual assumptions of lack of correlation. 
%
\begin{thm}[Lack of Identification]
  \label{pro:Lack}
  Suppose that Assumption \ref{assump:A2} holds and additionally that $\mathbb{E}[\varepsilon|T^*,T,z]=\mathbb{E}[\varepsilon|T^*,z]$ (non-differential measurement error).  
  Then regardless of how many values $z$ takes on, generically $\beta$ is unidentified based on the observables contained in Table \ref{tab:observables}.
\end{thm}
The preceding argument establishes lack of identification by deriving a parametric relationship between $\beta$ and $\alpha_0, \alpha_1, \{m^*_{1k}\}_{k=1}^K$.
So long as we adjust the other parameters according to this relationship, we are free to vary $\beta$ while leaving all observable moments unchanged.
This holds regardless of the number of values, $K$, that the instrument takes on.
