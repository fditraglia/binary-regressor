%!TEX root = ./main.tex
\section{The Case of a Continuous Instrument} 
I think we'll be able to say something helpful about the case in which the instrument is continuous rather than discrete.
For one, we can explain how one could discretize a continuous instrument provided that one is willing to assume that it is independent of the regression error term.

But it also looks like some of the equations we wrote down go through without modification when $z$ is continuous.
For example define the first-stage functions $\pi^*(z) = E(T^*|z)$ and $\pi(z) = E(T|z)$.
Then by precisely the same argument as given above in the section on identification by homoskedasticity,
\begin{equation}
  E[T^*|z] = \frac{E[T|z] - \alpha_0}{1 - \alpha_0 - \alpha_1}
\end{equation}
so the observable first-stage is just a shifted and scaled version of the ``true'' first-stage.
Proceeding similarly for the reduced form, we have 
\begin{equation}
  E[y|z] =  \alpha + \beta E[T^*|z]
\end{equation}
Substituting the expression for the first-stage gives
\begin{equation}
  E[y|z] = \alpha + \beta \left( \frac{E[T|z]- \alpha_0}{1 - \alpha_0 - \alpha_1} \right) 
\end{equation}
and since 
\begin{equation}
  E[y] = \alpha + \beta E[T^*] = \alpha + \beta \left( \frac{E[T] - \alpha_0}{1 - \alpha_0 - \alpha_1} \right)
\end{equation}
we see that
\begin{equation}
  \frac{E[y|z] - E[y]}{E[T|z] - E[T]} = \frac{\beta}{1 - \alpha_0 - \alpha_1}
\end{equation}
which is very similar to the expressions we wrote down before for IV but in this case involves an \emph{arbitrary} reduced form function $E[y|z]$ and an \emph{arbitrary} first-stage $E[T|z]$.
What is particularly interesting about the preceding expression is that it is \emph{just as informative} when $z$ is binary as when it takes on three values or is continuous, assuming the model is correct.
However it provides over-identifying information since we can evaluate the function at any value of $z$: we should get the same result in each case.

I wonder if we can proceed like this for all of the other conditions we use in our analysis. 
I also wonder whether a continuous instrument makes identification any easier.
Can we get by without the homoskedasticity assumption?
My guess is that we can since we could try to extend Lewbel to an instrument with more than three values, the answer is yes.

I think there's no avoiding the fact that we get identification from a non-linearity in the first stage, which is something that Angrist and Pishcke, for example, criticize in \emph{Mostly Harmless}.
For example, I think the Lewbel determinant condition must imply this.

It would be nice to link what we do with the Hausman et al work and also situations in which people use a first-stage probit, etc.
