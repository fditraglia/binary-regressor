%!TEX root = ./main.tex
\section{Lewbel (2007)}

Lewbel shows that under an exogenous (but missclasified) treatment,
and an instrument that takes on (at least) three values, the treatment
effect is identified. Lewbel's three-valued instrument equivalent
to having two binary instruments. Here we show how the logic in Lewbel's
argument maps into the two binary instruments case within our framework.

The model is 

\[
\mathbb{E}[Y|T^{*},T]=\alpha+\beta T^{*}
\]


Using iterated expectations over the distribution of $T^{*}$ given
$T$, 

\[
\mathbb{E}[Y|T]=\mathbb{E}_{T^{*}|T}\left[\mathbb{E}[Y|T^{*},T]\right]=\mathbb{P}(T^{*}=1|T)\mathbb{E}[Y|T^{*}=1]+\mathbb{P}(T^{*}=0|T)\mathbb{E}[Y|T^{*}=0]
\]
\[
\mathbb{E}[Y|T]=\mathbb{P}(T^{*}=1|T)(\alpha+\beta)+\mathbb{P}(T^{*}=0|T)\alpha
\]
\[
\mathbb{E}[Y|T]=\alpha+\mathbb{P}(T^{*}=1|T)\beta
\]
which implies that 
\[
\mathbb{E}[Y|T=1]-\mathbb{E}[Y|T=0]\equiv\beta^{OLS}=\left[\mathbb{P}(T^{*}=1|T=1)-\mathbb{P}(T^{*}=1|T=0)\right]\beta
\]


Lewbel defines $M(\alpha_{0},\alpha_{1},p)=\mathbb{P}(T^{*}=1|T=1)-\mathbb{P}(T^{*}=1|T=0)$,
implying that 
\begin{equation}
\beta=\frac{\beta^{OLS}}{M(\alpha_{0},\alpha_{1},p)}
\end{equation}


Lewbel also points out that $M(\alpha_{0},\alpha_{1},p)$ can also
be expressed as 
\[
M(\alpha_{0},\alpha_{1},p)=\frac{1}{1-\alpha_{0}-\alpha_{1}}\left[1-\frac{(1-\alpha_{1})\alpha_{0}}{p}-\frac{(1-\alpha_{0})\alpha_{1}}{1-p}\right]
\]
which I show below:

Using Bayes rule,

\[
M(\alpha_{0},\alpha_{1},p)=\frac{\mathbb{P}(T=1|T^{*}=1)p^{*}}{p}-\frac{\mathbb{P}(T=0|T^{*}=1)p^{*}}{1-p}
\]
\[
M(\alpha_{0},\alpha_{1},p)=\frac{(1-\alpha_{1})p^{*}}{p}-\frac{\alpha_{1}p^{*}}{1-p}
\]
\begin{equation}
=p^{*}\frac{(1-\alpha_{1})(1-p)-\alpha_{1}p}{p(1-p)}
\end{equation}


Recall that previously we have shown that

\[
p^{*}=\frac{p-\alpha_{0}}{1-\alpha_{0}-\alpha_{1}}
\]
Replacing for $p^{*}$ in equation $(2)$, 
\[
M(\alpha_{0},\alpha_{1},p)=\frac{p-\alpha_{0}}{1-\alpha_{0}-\alpha_{1}}\frac{(1-\alpha_{1})(1-p)-\alpha_{1}p}{p(1-p)}
\]
\[
=\frac{1}{1-\alpha_{0}-\alpha_{1}}\left[\frac{(p-\alpha_{0})(1-\alpha_{1})(1-p)-(p-\alpha_{0})\alpha_{1}p}{p(1-p)}\right]
\]
\[
=\frac{1}{1-\alpha_{0}-\alpha_{1}}\left[\frac{p(1-p)-p(1-p)\alpha_{1}-(1-p)(1-\alpha_{1})\alpha_{0}-(p-\alpha_{0})\alpha_{1}p}{p(1-p)}\right]
\]
\[
=\frac{1}{1-\alpha_{0}-\alpha_{1}}\left[1-\frac{(1-p)(1-\alpha_{1})\alpha_{0}+p\alpha_{1}-p^{2}\alpha_{1}+p^{2}\alpha_{1}-\alpha_{0}\alpha_{1}p}{p(1-p)}\right]
\]
\[
=\frac{1}{1-\alpha_{0}-\alpha_{1}}\left[1-\frac{(1-p)(1-\alpha_{1})\alpha_{0}}{p(1-p)}-\frac{p\alpha_{1}(1-\alpha_{0})}{p(1-p)}\right]
\]
\begin{equation}
=\frac{1}{1-\alpha_{0}-\alpha_{1}}\left[1-\frac{(1-\alpha_{1})\alpha_{0}}{p}-\frac{(1-\alpha_{0})\alpha_{1}}{1-p}\right]
\end{equation}


Now we use iterated expectations over the distribution of $T^{*}$
given $T$ and $Z_{k}$, where $Z_{k}$ are the instruments ($k=1,2$):
\[
\mathbb{E}[Y|T,Z_{k}]=\mathbb{E}_{T^{*}|T,Z_{k}}\left[\mathbb{E}[Y|T^{*},T,Z_{k}]\right]=\mathbb{P}(T^{*}=1|T,Z_{k})\mathbb{E}[Y|T^{*}=1]+\mathbb{P}(T^{*}=0|T,Z_{k})\mathbb{E}[Y|T^{*}=0]
\]
\[
\mathbb{E}[Y|T,Z_{k}]=\mathbb{P}(T^{*}=1|T,Z_{k})(\alpha+\beta)+\mathbb{P}(T^{*}=0|T,Z_{k})\alpha
\]
\[
\mathbb{E}[Y|T,Z_{k}]=\alpha+\mathbb{P}(T^{*}=1|T,Z_{k})\beta
\]
which implies that
\[
\mathbb{E}[Y|T=1,Z_{k}]-\mathbb{E}[Y|T=0,Z_{k}]\equiv\beta_{Z_{k}}^{OLS}=\left[\mathbb{P}(T^{*}=1|T=1,Z_{k})-\mathbb{P}(T^{*}=1|T=0,Z_{k})\right]\beta
\]


Notice that analogous to equation (3), 
\[
\mathbb{P}(T^{*}=1|T=1,Z_{k})-\mathbb{P}(T^{*}=1|T=0,Z_{k})\equiv M(\alpha_{0},\alpha_{1},p_{k})=\frac{1}{1-\alpha_{0}-\alpha_{1}}\left[1-\frac{(1-\alpha_{1})\alpha_{0}}{p_{k}}-\frac{(1-\alpha_{0})\alpha_{1}}{1-p_{k}}\right]
\]
where $p_{k}=\mathbb{P}(T=1|Z_{k})$, under the assumption that the
missclasiffication probabilities are independent of $Z_{k}$ which
Lewbel assumes.

This implies that if we run an OLS regression on the observations
for which $Z_{k}=1$ only, then we have that 
\begin{equation}
\beta=\frac{\beta_{Z_{k}=1}^{OLS}}{M(\alpha_{0},\alpha_{1},p_{k})}
\end{equation}
Since we have two instruments, we have two of these equations. Equations
(1) and (4) imply that 
\[
\frac{\beta^{OLS}}{M(\alpha_{0},\alpha_{1},p)}=\frac{\beta_{Z_{k}=1}^{OLS}}{M(\alpha_{0},\alpha_{1},p_{k})}
\]
\[
\beta^{OLS}M(\alpha_{0},\alpha_{1},p_{k})=\beta_{Z_{k}=1}^{OLS}M(\alpha_{0},\alpha_{1},p)
\]


\begin{equation}
\beta^{OLS}M(\alpha_{0},\alpha_{1},p_{k})-\beta_{Z_{k}=1}^{OLS}M(\alpha_{0},\alpha_{1},p)=0,\qquad k=1,2
\end{equation}


(5) are two equations in two unknowns, $\alpha_{0}$ and $\alpha_{1}$:

\[
\beta^{OLS}\frac{1}{1-\alpha_{0}-\alpha_{1}}\left[1-\frac{(1-\alpha_{1})\alpha_{0}}{p_{k}}-\frac{(1-\alpha_{0})\alpha_{1}}{1-p_{k}}\right]-\beta_{Z_{k}=1}^{OLS}\frac{1}{1-\alpha_{0}-\alpha_{1}}\left[1-\frac{(1-\alpha_{1})\alpha_{0}}{p}-\frac{(1-\alpha_{0})\alpha_{1}}{1-p}\right]=0
\]
\[
\beta^{OLS}\left[1-\frac{(1-\alpha_{1})\alpha_{0}}{p_{k}}-\frac{(1-\alpha_{0})\alpha_{1}}{1-p_{k}}\right]-\beta_{Z_{k}=1}^{OLS}\left[1-\frac{(1-\alpha_{1})\alpha_{0}}{p}-\frac{(1-\alpha_{0})\alpha_{1}}{1-p}\right]=0
\]
\[
(1-\alpha_{1})\alpha_{0}\left[\frac{\beta_{Z_{k}=1}^{OLS}}{p}-\frac{\beta^{OLS}}{p_{k}}\right]+(1-\alpha_{0})\alpha_{1}\left[\frac{\beta_{Z_{k}=1}^{OLS}}{1-p}-\frac{\beta^{OLS}}{1-p_{k}}\right]=\beta_{Z_{k}=1}^{OLS}-\beta^{OLS}
\]
which we can rewrite as 
\[
B_{0}w_{0}^{k}+B_{1}w_{1}^{k}=w_{2}^{k}
\]


This is a system of two linear equations in two unknowns, $B_{0}$
and $B_{1}$. In matrix form,

\[
\left[\begin{array}{cc}
w_{0}^{1} & w{}_{1}^{1}\\
w_{0}^{2} & w_{1}^{2}
\end{array}\right]\left[\begin{array}{c}
B_{0}\\
B_{1}
\end{array}\right]=\left[\begin{array}{c}
w_{2}^{1}\\
w_{2}^{2}
\end{array}\right]
\]


as long as $w_{0}^{1}w_{1}^{2}-w_{0}^{2}w_{1}^{1}\neq0$ (which is
Assumption A5 in Lewbel (2007)),

\[
\left[\begin{array}{c}
B_{0}\\
B_{1}
\end{array}\right]=\frac{1}{w_{0}^{1}w_{1}^{2}-w_{0}^{2}w_{1}^{1}}\left[\begin{array}{c}
w_{1}^{2}w_{2}^{1}-w_{1}^{1}w_{2}^{2}\\
w_{0}^{1}w_{2}^{2}-w_{0}^{2}w_{2}^{1}
\end{array}\right]
\]


Finally, given that $B_{0}=(1-\alpha_{1})\alpha_{0}$ and $B_{1}=(1-\alpha_{0})\alpha_{1}$,
we can solve for the missclassification rates:

\[
B_{1}=\alpha_{1}-\alpha_{1}\frac{B_{0}}{1-\alpha_{1}}
\]
\[
\alpha_{1}=\frac{1}{2}\left[1-B_{0}+B_{1}\pm\sqrt{(1-B_{0}+B_{1})^{2}-4B_{1}}\right]
\]


and 
\[
\alpha_{0}=\frac{B_{0}}{1-\frac{1}{2}\left[1-B_{0}+B_{1}\pm\sqrt{(1-B_{0}+B_{1})^{2}-4B_{1}}\right]}
\]

Once we have $(\alpha_{0},\alpha_{1})$ we can go back to equation
$(1)$ and recover $\beta$.

In page 544 Lewbel observes that if hs instrument were binary (if
we only had one instrument in our case), identification could be achieved
with one additional restriction on the missclasification rates. One
such restriction is implied by homoskedasticity on the instrument,
which he does not mention. 
