%!TEX root = ./main.tex
\section{Introduction}

This paper studies the use of a valid instrument to identify the causal effect of an endogeneous, binary treatment that is subject to non-differential measurement error in a non-parametric regression model with additively separable errors. 
Although a relevant case for applied work, this setting has received little attention in the literature.
The only existing result appears in an important paper by \cite{Mahajan}, who proves identification by relying on a discrete instrument that takes on at least two values.   
Here we show that this result is incorrect. 
To do so, we begin by providing a convenient notational framework within which to situate the problem.
Using this framework we show that the proof in Appendix A.2 of \cite{Mahajan} leads to a contradiction.
Throughout the paper, \cite{Mahajan} maintains an assumption (Assumption 4) which he calls the ``Dependency Condition.'' 
This assumption requires that the instrumental variable be relevant.
When extending his result for an exogenous treatment to the more general case of an endogenous one in a model with additively separable errors, however, he must impose an additional condition on the model (Equation 11), which turns out to imply the lack of a first-stage, violating the Dependency Condition.

Since one cannot impose the condition in Equation 11 of \cite{Mahajan}, we go on to study the prospects for identification in this model more broadly.
We consider two possibilities.
First, borrowing an idea from \cite{Lewbel}, we explore whether expanding the support of the instrument, so that it takes on more than two values, yields identification.
Allowing the instrument to take on additional values increases the number of available moment conditions.
We show, however, that these additional moments cannot point identify the treatment effect.
This holds true regardless of how many (finite) values the instrument takes on.

We then consider a new source of identifying information in the form of a conditional homoskedasticity assumption. 
In particular, we suppose that the conditional \emph{variance} of the regression error term given the instrument is constant.
While stronger than the usual mean independence assumption, this assumption holds automatically in a randomized controlled trial or a genuine natural experiment. 
To the best of our knowledge, this source of information has not been exploited in the extant literature on instrumental variables.  
We show that this assumption leads to a novel partial identification result that is easy to implement in practice and can be applied regardless of the number of values that the instrument takes on.
Moreover, it can be used to obtain point identification in some special cases that nevertheless may be empirically relevant. 

The remainder of this paper is organized as follows. In section 2 we discuss the literature in relation to our framework. Section 3 then lays out the econometric model, its assumptions, and our notational framework. Section 4 then presents two of our main results: A proof of \citeauthor{Mahajan}'s incorrect claim of identification in under non-differential misclassification with a valid instrument and an endogenous treatment, and our proof of the generic un-identification of this model. Section 5 then presents our partial identification results relying on a conditional homoskedasticity assumption, a test for the presence of misclassification, and a simulation exercise illustrating its usefulness. Section 6 concludes.


 


