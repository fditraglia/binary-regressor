%!TEX root = ./main.tex
\section{Introduction}

Measurement error and endogeneity are pervasive features of economic data.
Conveniently, a valid instrumental variable corrects for both problems when the measurement error is classical, i.e.\ uncorrelated with the truth.
Many regressors of interest in applied work, however, are binary and thus cannot be subject to classical measurement error.\footnote{The only way to mis-classify a true one is downwards, as a zero, while the only way to mis-classify a true zero is upwards, as a one. This creates negative dependence between the truth and the error.}
When faced with non-classical measurement error, the instrumental variables estimator can be severely biased.
In this paper, we study an additively separable model of the form
\begin{equation}
  y = c(\mathbf{x}) + \beta(\mathbf{x}) T^* + \varepsilon
  \label{eq:model}
\end{equation}
where $\varepsilon$ is a mean-zero error term, $T^*$ is a binary regressor of interest and $\mathbf{x}$ is a vector of exogenous controls.\footnote{Because $T^*$ is binary, there is no loss of generality from writing the model in this form rather than the more familiar $y = h(T^*,\mathbf{x})+\varepsilon$. Simply define $\beta(\mathbf{x}) = h(1,\mathbf{x}) - h(0,\mathbf{x})$ and $c(\mathbf{x}) = h(0,\mathbf{x})$.}
Our question is whether, and if so under what conditions, a discrete instrumental variable $z$ suffices to non-parametrically identify the causal effect $\beta(\mathbf{x})$ of $T^*$, when we observe not $T^*$ itself but a mis-classified binary surrogate $T$. 

We proceed under the assumption of non-differential measurement error.
This condition has been widely used in the existing literature and imposes, roughly speaking, that $T$ provides no additional information beyond that contained in $(T^*,\mathbf{x})$.
Even in this fairly standard setting, identification remains an open question.
We begin by showing that the only existing identification result for this model is incorrect.
We then go on to derive the sharp identified set under the standard first-moment assumptions from the related literature.
We show that regardless of the number of values that $z$ takes on, the model is not point identified.
This motivates us to consider alternative, and slightly stronger assumptions.
We show that, given a binary instrument, the addition of a second moment independence assumption suffices to identify a model with one-sided mis-classification.
Adding a second moment restriction on the measurement error along with a third moment independence assumption for the instrument suffices to identify the model in general.
This result likewise requires only a binary $z$.

We then turn our attention to inference, showing that both our model and related models from the literature that assume an exogenous regressor suffer from a weak identification problem. 
In essence, binary mis-classification creates a mixture model and to correct the bias in the instrumental variables estimator, we must estimate the mixing probabilities.
But when $\beta(\mathbf{x})$ is small the ``mixture modes'' are nearly indistinguishable, so that it becomes impossible to reliably estimate these probabilities.
To address this difficulty, we exploit the inequality moment restrictions that emerge from our derivation of the sharp identified set.
These restrictions remain informative even when $\beta(\mathbf{x})$ is small or zero.
Combining them with the moment equalities that emerge from our identification result, we propose a robust inference procedure using tools from the moment inequality literature.
Our procedure is computationally attractive and performs well in simulations.
Moreover, it can be used both in our model and related models from the literature that assume an exogenous $T^*$. 

Our work relates to a large literature studying departures from the textbook linear, classical measurement error setting.
One strand of this literature considers relaxing the assumption of linearity while maintaining that of classical measurement error.
\cite{schennach2004}, for example, uses repeated measures of each mis-measured regressor to obtain identification, while \cite{schennach2007} uses an instrumental variable.  
More recently, \cite{SongSchennachWhite} rely on a repeated measure of the mis-measured regressor and the existence of a set of additional regressors, conditional upon which the regressor of interest is unrelated to the unobservables, to obtain identification.      
For comprehensive reviews of the challenges of addressing measurement error in non-linear models, see  \cite{chensurvey} and \cite{SchennachSurvey}.
Another strand of the literature considers relaxing the assumption of classical measurement error, by allowing the measurement error to be related to the true value of the unobserved regressor.
\cite{ChenHongTamer} obtain identification in a general class of moment condition models with mis-measured data by relying on the existence of an auxiliary dataset from which they can estimate the measurement error process.
In contrast, \cite{HuSchennach} and \cite{song2015} rely on an instrumental variable and an additional conditional location assumption on the measurement error distribution. 
More recently, \cite{HuShiuWoutersen} use a continuous instrument to identify the ratio of partial effects of two continuous regressors, one measured with error, in a linear single index model.
Unfortunately, these approaches cannot be applied to the case of a mis-measured binary regressor.

A number of papers have studied models with an exogenous binary regressor subject to non-differential measurement error.\footnote{As defined above, non-differential measurement error assumes that $T$ provides no additional information beyond that already contained in $(T^*,\mathbf{x})$.}
One group of papers asks what can be learned without recourse to an instrumental variable.
An early contribution by \cite{Aigner} characterizes the asymptotic bias of OLS in this setting, and proposes a correction using outside information on the mis-classification process.
Related work by \cite{Bollinger} provides partial identification bounds.
More recently, \cite{ChenHuLewbel} use higher moment assumptions to obtain identification in a linear model, and \cite{ChenHuLewbel2} extend these results to the non-parametric setting. 
\cite{HasseltBollinger} and \cite{BollingerHasseltWP} provide additional partial identification results.
For results on the partial identification of discrete probability distributions under mis-classification, see \cite{molinari}.

Continuing under the assumption of exogeneity and non-differential measurement error, another group of papers relies on the availability of either an instrumental variable or a second measure of $T^*$.
\cite{BBS} and \cite{KRS} consider a linear model and show that when \emph{two} alternative measures $T_1$ and $T_2$ of $T^*$ are available, a non-linear GMM estimator can be used to recover the effect of interest.
Subsequently, \cite{FL} note that an instrumental variable can take the place of one of the measures.
\cite{Mahajan} extends the results of \cite{BBS} and \cite{KRS} to a more general setting using a binary instrument in place of one of the treatment measures, establishing non-parametric identification of the conditional mean function.
When the $T^*$ is in fact exogenous, this coincides with the causal effect.
\cite{hu2008} derives related results when the mis-classified discrete regressor may take on more than two values.
\cite{Lewbel} provides an identification result for the same model as \cite{Mahajan} under different assumptions.
In particular, the variable that plays the role of the ``instrument'' need not satisfy the exclusion restriction provided that it does not interact with the treatment and takes on at least three distinct values. 

Much less is known about the case in which a binary, or discrete, regressor is not only mis-measured but endogenous.
The first paper to provide a formal result for this case is \cite{Mahajan}.
He extends his main result to the case of an endogenous treatment, providing an explicit proof of identification under the usual IV assumption in a model with additively separable errors.
As we show below, however, \citeauthor{Mahajan}'s proof is incorrect.

\todo[inline]{Update from here down}

Unlike \cite{kreider2012}, who partially identify the effects of food stamps on health outcomes of children under weak measurement error assumptions, we do not rely on auxiliary data. 
Unlike \cite{shiu2015}, who considers a sample selection model with a discrete, mis-measured, endogenous regressor, we do not rely on a parametric assumption about the form of the first-stage.
Finally unlike \cite{Ura}, who studies local average treatment effects without assuming non-differential measurement error but presents only partial identification results, we 

point identify treatment effect under non-differential measurement error.

Moreover, unlike the identification strategies from the existing literature described above, we do not rely upon continuity of the instrument, a large support condition, or restrictions on the relationship between the true, unmeasured treatment and its observed surrogate, subject to the condition that the measurement error process is non-differential.


\todo[inline]{Some updated comments on related papers}
\paragraph{How are we different from Ura?}
\begin{enumerate}
  \item We maintain nondifferential measurement error assumption throughout; Ura's main purpose is to relax it.
  \item We focus on an additively separable model; Ura explicitly studies at LATE setting
  \item We obtain point and partial identification results; Ura present only partial identification results
\end{enumerate}

\paragraph{What about Denteh et al?}
\cite{nguimkeu2016estimation} parametric model of endogenous participation with one-sided endogenous mis-reporting (false negatives).
True participation is a normal linear threshold model with covariates $z$ and parameter $\gamma$, and mis-reporting is also a normal linear threshold model with covariates $w$ and parameter $\theta$.
Requires $z$ to be exogenous wrt to all error terms (essentially an instrument) but do not require $w$ to be exogenous.
Need $z$ and $w$ to be different, although there can be overlap.
Definitely a parametric model, although they say normality isn't needed you do need to specify the joint distribution of the error terms.
Requires exclusion restrictions for both participation and measurement error.
Basically they're using parametric assumptions plus exclusion restrictions to get identification in a one-sided mis-classification model.
In contrast we study the problem non-parametrically and allow mis-reporting to depend arbitrarily on $\mathbf{x}$.

Should say that we also relate to a large literature on moment inequalities.
We use the GMS procedure from Andrews and Soares.
And though we don't take the approach that they do, our idea of combining the inequalities with equalities is related to Moon and Schorfheide.

\todo[inline]{Change from here UPWARDS}

The remainder of the paper is organized as follows.
Section \ref{sec:baseline} describes our model and assumptions, Section \ref{sec:ident_literature} relates our results to existing work, and Sections \ref{sec:partial}--\ref{sec:point} present our identification results.
Section \ref{sec:problem} points out the special inferential difficulties that arise in models with mis-classification while Section \ref{sec:overview} gives a high-level overview of our proposed inference procedure.
Full details of the procedure follow in Sections \ref{sec:inequalities}--\ref{sec:details}.
Section \ref{sec:simulation} presents simulation results, and Section \ref{sec:conclusion} concludes.
Proofs appear in Appendix \ref{sec:proofs}, and we give a detailed explanation of the error in \cite{Mahajan} in Appendix \ref{sec:mahajan}.


