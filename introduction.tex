%!TEX root = ./main.tex
\section{Introduction}

This paper studies the use of a discrete instrumental variable to identify the causal effect of a endogenous, mis-measured, binary treatment in a homogeneous effects model with additively separable errors.
Although a relevant case for applied work, this setting has received little attention in the literature.
The only existing result for the case of an endogenous treatment appears in an important paper by \cite{Mahajan}, who is primarily concerned with the case of an exogenous treatment.
As we show below, \citeauthor{Mahajan}'s identification result for the endogenous treatment case is incorrect.
As far as we are aware, this leaves the problem considered in this paper completely unsolved.

We begin by providing a convenient notational framework within which to situate the problem.
Using this framework we then show that the proof in Appendix A.2 of \cite{Mahajan} leads to a contradiction.
Throughout his paper, \cite{Mahajan} maintains an assumption (Assumption 4) which he calls the ``Dependency Condition.'' 
This assumption requires that the instrumental variable be relevant.
When extending his result for an exogenous treatment to the more general case of an endogenous one, however, he must impose an additional condition on the model (Equation 11), which turns out to imply the lack of a first-stage, violating the Dependency Condition.

Since one cannot impose the condition in Equation 11 of \cite{Mahajan}, we go on to study the prospects for identification in this model more broadly.
We consider two possibilities.
First, since \citeauthor{Mahajan}'s identification results require only a binary instrument,  we borrow an idea from \cite{Lewbel} and explore whether expanding the support of the instrument yields identification based on moment equations similar to those used by \cite{Mahajan}.
While allowing the instrument to take on additional values does increase the number of available moment conditions, we show that these moments cannot point identify the treatment effect, regardless of how many (finite) values the instrument takes on.

We then consider a new source of identifying information that arises from imposing stronger assumptions on the instrumental variable.
\cite{Mahajan} and related papers, discussed below, use only conditional means of the outcome to identify the treatment effect.
However, if the instrument is not merely mean independent but in fact \emph{statistically independent} of the regression error term, as in a randomized controlled trial or a true natural experiment, additional moment conditions become available.
To the best of our knowledge, this source of information has not been exploited in the extant literature on instrumental variables.  
Under this stronger condition on the instrument, we first show that conditional second moments of the outcome variable identify the \emph{difference} of mis-classification rates in the mis-measured regressor: the probability that a true one is classified as a zero minus the probability that a true zero is classified as a one.
Because these rates must equal each other when there is no mis-classification error, our result can be used to test a necessary condition for the absence of measurement error.   
It can also be used to construct simple and informative partial identification bounds for the treatment effect.
When one of the mis-classification rates is known, this identifies the treatment effect.
More generally, however, this is not the case.
We go on to show that conditional third moments point identify one of the mis-classification rates.
Thus, combining conditional first, second and third moment information point identifies the treatment effect.

The remainder of this paper is organized as follows. 
In section 2 we discuss the literature in relation to the problem considered here. 
Section 3 then lays out the econometric model, its assumptions, and our notational framework. 
Section 4 considers identification based on conditional means, showing that \citeauthor{Mahajan}'s proof is incorrect and that increasing the support of the instrument cannot be used to obtain identification.
Section 5 considers presents our results under stronger conditions on the instrument, based on conditional second and third moments of the outcome variable.
Section 6 concludes.  

 


