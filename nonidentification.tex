%!TEX root = ./main.tex

\subsection{Generic Lack of Identification}
The full set of parameters needed to characterize the model in Equation \ref{eq:linear} consists of $\beta, \alpha_0, \alpha_1$ and the conditional means of $u$, namely $m^*_{tk}$ for a total of $2K+3$ parameters.
In contrast, there are only $2K$ available moment conditions, namely:
\begin{align}
  \label{eq:MC0}
  \hat{y}_{0k} &=\frac{\alpha_1(p_k - \alpha_0)(\beta + m_{1k}^*) + (1 - \alpha_0)(1 - p _k -  \alpha_1)m_{0k}^*}{1 - \alpha_0 - \alpha_1} \\[1.5ex]
  \label{eq:MC1}
  \hat{y}_{1k} &= \frac{(1-\alpha_1)(p_k - \alpha_0)(\beta+m_{1k}^*)+  \alpha_0(1-p_k - \alpha_1)m_{0k}^*}{1-\alpha_0 - \alpha_1}
\end{align}
by the Law of Iterated Expectations, where the observables on the left hand side are defined according to $\hat{y}_{0k} = (1-p_k)\bar{y}_{0k}$ and $\hat{y}_{1k}= p_k \bar{y}_{1k}$.
Notice that the observable ``weighted'' cell mean $\hat{y}_{tk}$ depends on both $m^*_{tk}$ \emph{and} $m^*_{1-t,k}$ since the cell in which $T=t$ from Table \ref{tab:observables} is in fact a mixture of both the cells $T^*=0$ and $T^*=1$ from Table \ref{tab:unobservables}, for a particular column $k$.

Clearly we have fewer equations than unknowns.
What additional restrictions could we consider imposing on the system? 
In a very interesting paper, \cite{Lewbel} proposes using a three-valued ``instrument'' that does \emph{not} satisfy the exclusion restriction.
By assuming instead that there is no \emph{interaction} between the instrument and the treatment, he is able to prove identification of the treatment effect.
Using our notation it is very easy to see why and how \citeauthor{Lewbel}'s argument works.
His moment conditions are equivalent to Equations \ref{eq:MC0} and \ref{eq:MC1} with the additional restriction that $m^*_{0k} = m^*_{1k}$ for all $k= 1, \dots, K$.
This leaves the number of equations unchanged at $2K$, but reduces the number of unknowns to $K+3$.
The smallest $K$ for which $K+3$ is at least as large as $2K$ is 3, which makes it clear why \citeauthor{Lewbel}'s proof must require that the ``instrument'' take on at least three values.\footnote{The context considered by \cite{Lewbel} is slightly different from the one we pursue here, in that his ``instrument'' is more like a covariate: it is allowed to have a direct effect on the outcome of interest.  For this reason, \cite{Lewbel} cannot use the exogeneity of the treatment to obtain identification based on a two-valued instrument.}
%Footnote explaining why Lewbel can't use a regressor exog condition to get identification with a two-valued instrument: it's because he's leaving out a relevant regressor, namely z. This means that the error term of changes. See our whiteboard photo from 2015-09-18 14.04.59

Unlike \cite{Lewbel}, we, along with \cite{Mahajan} and others, assume that $z$ satisfies the exclusion restriction. 
This implies a different constraint on the $m^*_{tk}$ from Table \ref{tab:unobservables}.


Using Equation \ref{eq:pkstar} and rearranging gives 
\begin{equation*}
  \frac{(1 - p_k - \alpha_1) m_{0k}^*}{1 - \alpha_0 - \alpha_1} = c - \frac{(p_k - \alpha_0)m_{1k^*}}{1 - \alpha_0 - \alpha_1}
\end{equation*}
which we can substitute into Equations \ref{eq:MC0} and \ref{eq:MC1} to yield
\begin{align}
  \label{eq:MC0IV}
  \hat{y}_{0k} &=\alpha_1(p_k - \alpha_0)\left(\frac{\beta}{1 - \alpha_0 - \alpha_1}\right) + (1-\alpha_0)c - (p _k -  \alpha_0)m_{1k}^* \\[1.5ex]
  \label{eq:MC1IV}
  \hat{y}_{1k} &=(1-\alpha_1)(p_k - \alpha_0)\left(\frac{\beta}{1 - \alpha_0 - \alpha_1}\right) + \alpha_0 c + (p _k -  \alpha_0)m_{1k}^*.
\end{align}
Equations \ref{eq:MC0IV} and \ref{eq:MC1IV} also make it clear why the IV estimator is inconsistent in the face of non-differential measurement error, and that this inconsistency does not depend on the endogeneity of the treatment, as noted by \cite{FL}.
Adding together Equations \ref{eq:MC0IV} and \ref{eq:MC1IV} yields
\begin{equation*}
  \hat{y}_{0k} + \hat{y}_{1k} = c + (p_k - \alpha_0)\left( \frac{\beta}{1 - \alpha_0 - \alpha_1} \right) 
\end{equation*}
completely eliminating the $m^*_{1k}$ from the system.
Taking the difference of the preceding expression expression evaluated at two different values of the instrument, $z_{k}$ and $z_{\ell}$, and rearranging
\begin{equation}
  \mathcal{W} = \frac{(\hat{y}_{0k} + \hat{y}_{1k}) - (\hat{y}_{0\ell} + \hat{y}_{1\ell})}{p_k - p_\ell} =  \frac{\beta}{1 - \alpha_0 - \alpha_1}
  \label{eq:wald}
\end{equation}
which is the well-known Wald IV estimator, since $\hat{y}_{0k} + \hat{y}_{1k} = \mathbb{E}[y|z = z_k]$.

Imposing that $\mathbb{E}[\varepsilon|z]=0$ replaces the $K$ unknown parameters $\left\{ m^*_{0k}\right\}_{k=1}^K $ with a single parameter $c$, leaving us with the same $2K$ equations but only $K+4$ unknowns.
When $K=2$ (a binary instrument) we have 4 equations and 6 unknowns.
So how can one identify $\beta$ in this case?
The literature has imposed additional assumptions which, using our notation, can once again be mapped into restrictions on the $m_{tk}^*$.
\cite{BBS}, \cite{KRS}, and \cite{Mahajan} make a \emph{joint} exogeneity assumption on $(T^*,z)$, namely $\mathbb{E}[\varepsilon|T^*,z]=0$.
Notice that this is strictly stronger than assuming that the instrument is valid and the treatment is exogenous.
In our notation, this joint exogeneity assumption is equivalent to imposing $m_{tk}^*=c$ for all $t,k$.
This reduces the parameter count to 4 regardless of the value of $K$.
Thus, when the instrument is binary, we have exactly as many equations as unknowns.
The arguments in \cite{BBS}, \cite{KRS}, and \cite{Mahajan} are all equivalent to solving Equations \ref{eq:MC0IV} and \ref{eq:MC1IV} for $\beta$ under the added restriction that $m^*_{1k}=c$, establishing identification for this case.
\cite{FL} use the same argument in a linear model with a potentially continuous instrument, but impose only the weaker conditions that the treatment is exogenous and the instrument is valid. 
Nevertheless, a crucial step in their derivation implicitly assumes the stronger joint exogeneity assumption used by \cite{BBS}, \cite{KRS} and \cite{Mahajan}.
Without this assumption, their proof does not in fact go through.
%Add a footnote showing the gap between the two sets of assumptions and how it relates to the continuous case, triple moment, etc.

If one wishes to allow for an endogenous treatment, clearly the joint exogeneity assumption $m_{tk}^*=c$ is unusable: we are back to $2K$ equations in $K+4$ unknowns.
Based on the identification arguments described above, there would seem to be two possible avenues for identification of the treatment effect when a valid instrument is available.
A first possibility would be to impose alternative conditions on the $m^*_{tk}$ that are compatible with an endogenous treatment.
If $z$ is binary, two additional restrictions would suffice to equate the counts of moments and unknowns.
This is the route followed by \cite{Mahajan} in his proof of identification with a binary instrument and endogenous treatment.
His Equation (11), expressed in our notation, amounts to adding two cross-column restrictions in Table \ref{tab:unobservables}: $m^*_{11} = m^*_{12}$ and $m^*_{01} = m^*_{02}$. 
Another possibility, suggested by \citeauthor{Lewbel}'s approach, would be to rely on an instrument that takes on more than two values.
Following this approach would suggest a 4-valued instrument, the smallest value of $K$ for which $2K = K+4$.
In the following section we present two of our main results: first \citeauthor{Mahajan}'s approach leads to a contradiction, and second, regardless of the value of $K$, $\beta$ is unidentified based on conditional mean information.



%\cite{FL} point out that an instrument can be used in place of a second measure of $T^*$ provided that $T^*$ is still exogenous.
%Essentially the same estimator as in \cite{BBS} and \cite{KRS} but more general since the instrument need not be binary: can in fact be continuous.
%But they make a mistake. 
%They assume $E[zu]=0$, $E[T^*u]=0$ and non-differential measurement error and claim that this is sufficient to consistently estimate $\beta$.
%However, this is incorrect: we need the additional assumption that $E[zT^*u]=0$ which is stronger.
%(They seem to think that this term only appears when you have an endogenous $T^*$.)
%While \cite{FL} are aware that there are some differences between two measures of $T^*$, as in \cite{BBS}, and an arbitrary instrument $z$, they seem to have missed one subtle point.
%The assumptions in \cite{BBS} in fact imply that $E[u|T^*,z]=0$.\footnote{This follows from Assumptions A1 and A2 combined with Equation 3.}
%From this it follows that $E[zT^*u]=E[zu]=E[T^*u]=0$. 
%However, if one takes the non-differential measurement assumption literally it is in fact sufficient in the case of two measures to assume only that $E[zu]=E[T^*u]=0$:
%\begin{align}
%  E[zT^*u] &= E[(T^*+w)T^*u] = E[(T^*)^2u] + E[wuT^*]  \\
%  &= E\left[ E\left( u|T^* \right)(T^*)^2 \right] + E\left[E\left( wu|T^* \right)T^*\right]\\
%  &= 0 + E\left[ E(w|T^*)E(u|T^*)T^* \right] = 0
%\end{align}
%using the fact that $E[u|T^*]=0$ and $w$ is independent of $u$ conditional on $T^*$.
%This argument does \emph{not} necessarily apply to an arbitrary instrument $z$: $E[zu]=E[T^*u]=0$ does not imply that $E[zT^*u]=0$.
%\todo[inline]{Put in our simple binary example.}
%While it might seem strange to assume in practice that $E[zu]=E[T^*u]=0$ are exogenous but not that $E[zT^*u]=0$ the point is merely that this is an additional assumption beyond the usual assumptions of lack of correlation. 
%
%
%Mahajan uses the assumption $E[u|T^*,z]$ to get identification using same basic estimator as \cite{BBS}.
%
%Another closely related paper is \cite{Lewbel}.
%Whereas Mahajan makes sufficient assumptions to identify $\beta$ with a two-valued $z$, provided that $T^*$ is exogenous, Lewbel works with a three-valued $z$.
%While Lewbel also assumes that $E[T^*u]=0$, his ``instrument'' is really more like a covariate.
%He assumes that $z$ is unrelated to the mis-classification probabilities but allows it to have a direct effect on $y$, as long as there is no interaction between $T^*$ and $z$.
%Since this involves imposing fewer restrictions on the $m_{ij}$, Lewbel requires that $z$ take on more values.
%There is also some kind of determinant condition that we don't fully understand yet, but will figure out soon!
%
%
%
We have seen that \cite{Mahajan}'s approach cannot identify $\beta$ when the treatment is endogenous: Assumption \ref{assump:Eq11} in fact implies that the instrument is \emph{irrelevant}. 
But this alone does not establish that a valid instrument is insufficient to identify $\beta$ when the treatment is endogenous.
In particular, our equation counts from above appear to suggest that a valid instrument that takes on at least four values might suffice for identification.
Unfortunately, this is not the case as we now show.

\begin{thm}[Lack of Identification]
  \label{pro:Lack}
  Suppose that Assumption \ref{assump:A2} holds and additionally that $\mathbb{E}[\varepsilon|T^*,T,z]=\mathbb{E}[\varepsilon|T^*,z]$ (non-differential measurement error).  
  Then regardless of how many values $z$ takes on, generically $\beta$ is unidentified based on the observables contained in Table \ref{tab:observables}.
\end{thm}
\begin{proof}[Proof of Theorem \ref{pro:Lack}]
  The assumptions of this Theorem are the same as those used to derive Equations \ref{eq:MC0IV} and \ref{eq:MC1IV}. 
  These expressions, for $k = 1, \dots, K$ constitute the full set of available moment conditions.
  To establish lack of identification, we derive a parametric relationship between $\beta$ and the other model parameters such that, varying $\beta$ along this parametric relationship, the observables $(\hat{y}_{0k},\hat{y}_{1k})$ are unchanged for all $k$.   

  Recall from the discussion preceding Equation \ref{eq:wald} that the Wald estimator $\mathcal{W} = \beta/(1-\alpha_0-\alpha_1)$ is identified in this model so long as $K$ is at least 2. 
  Rearranging, we find that:
  \begin{align*}
    \alpha_0 &= (1-\alpha_1) - \beta/\mathcal{W} \\
    (p_k - \alpha_0) &= p_k - (1-\alpha_1) + \beta/\mathcal{W}\\
    1 - \alpha_0 &= \alpha_1 + \beta/\mathcal{W}
  \end{align*}
Substituting these into Equations \ref{eq:MC0IV} and \ref{eq:MC1IV} and summing the two, we find, after some algebra, that
\[\hat{y}_{0k} + \hat{y}_{1k} + \mathcal{W}(1-p_k) = c + \beta + \mathcal{W} \alpha_1.\]
Since the left-hand side of this expression depends only on observables and the identified quantity $\mathcal{W}$, this shows that the right-hand side is itself identified in this model.
For simplicity, we define $\mathcal{Q} = c + \beta + \mathcal{W}\alpha_1$.
Since $\mathcal{W}$ and $\mathcal{Q}$ are both identified, varying either \emph{necessarily} changes the observables, so we must hold both of them constant. 
We now show that Equations \ref{eq:MC0IV} and \ref{eq:MC1IV} can be expressed in terms of $\mathcal{W}$ and $\mathcal{Q}$.
Conveniently, this eliminates $\alpha_0$ from the system.
After some algebra, 
\begin{align}
  \label{eq:OtherSameEquation}
  \hat{y}_{0k} &= \alpha_1 (\mathcal{Q} - m^*_{1k}) + \beta(c-m^*_{1k})/\mathcal{W} + (1-p_k)\left[m^*_{1k} - \mathcal{W}\alpha_1\right]\\
  \hat{y}_{1k} &= (1-\alpha_1) \mathcal{Q} + \beta(m^*_{1k} - c)/\mathcal{W} - (1-p_k)\left[m^*_{1k} + \mathcal{W}(1-\alpha_1)\right]
  \label{eq:SimplifyMe}
\end{align}
Now, rearranging Equation \ref{eq:SimplifyMe} we see that
\begin{equation}
  \mathcal{Q} - \hat{y}_{1k} - \mathcal{W}(1-p_k) = \alpha_1 (\mathcal{Q} - m^*_{1k}) + \beta(c-m^*_{1k})/\mathcal{W} + (1-p_k)\left[m^*_{1k} - \mathcal{W}\alpha_1\right]
  \label{eq:SameEquation}
\end{equation}
Notice that the right-hand side of Equation \ref{eq:SameEquation} is the \emph{same} as that of Equation \ref{eq:OtherSameEquation} and that $\mathcal{Q} - \hat{y}_{1k} - \mathcal{W}(1-p_k)$ is precisely $\hat{y}_{0k}$.
In other words, given the constraint that $\mathcal{W}$ and $\mathcal{Q}$ must be held fixed, we only have \emph{one} equation for each value that the instrument takes on.
Finally, we can solve this equation for $m^*_{1k}$ as
\begin{equation}
  m^*_{1k} = \frac{\mathcal{W}(\hat{y}_{0k}-\alpha_1 \mathcal{Q}) - \beta(\mathcal{Q}-\beta-\mathcal{W}\alpha_1) + \mathcal{W}^2(1-p_k)\alpha_1}{\mathcal{W}(1-p_k - \alpha_1) - \beta}
  \label{eq:mSolve}
\end{equation}
using the fact that $c = \mathcal{Q} - \beta - \mathcal{W}\alpha_1$.
Equation \ref{eq:mSolve} is a manifold parameterized by $(\beta,\alpha_1)$ that is \emph{unique} to each value that the instrument takes on. 
Thus, by adjusting $\left\{ m^*_{1k} \right\}_{k=1}^K$ according to Equation \ref{eq:mSolve} we are free to vary $\beta$ while holding all observable moments fixed. 
\end{proof}

%Notice that Equation 20 breaks if $p_k = \alpha_0$. This is because it would reduce the pair of equations to an expression  kill the $m^*_{1k}$ term in the pair of equations for this particular value of $k$. This seems to be related to \cite{Hausman}

%Add some discussion of what this proof means, why there is a lack of identification intuitively, why adding more values for the instrument doesn't help, etc.
