%!TEX root = ./main.tex
\section{Identification Based on Higher Moments}
Having shown that the moment conditions from Table \ref{tab:observables} do not identify $\beta$ regardless of the value of $K$, we now consider exploiting the information contained in higher moments of $y$. 
When $z$ is not merely mean-independent but in fact \emph{statistically} independent of $\varepsilon$, as in a randomized controlled trial or a true natural experiment, the following assumptions hold automatically.
\begin{assump}[Second Moment Independence]
  $\mathbb{E}[\varepsilon^2|z]=\mathbb{E}[\varepsilon^2]$
  \label{assump:homosked}
\end{assump}
\begin{assump}[Third Moment Independence]
  $\mathbb{E}[\varepsilon^3|z]=\mathbb{E}[\varepsilon^3]$
  \label{assump:skew}
\end{assump}
%When combined with the usual IV assumption, $\mathbb{E}[u|z]=0$, Assumption \ref{assump:homosked} implies that $Var(\varepsilon|z) = Var(\varepsilon)$.
%Whether this assumption is reasonable, naturally, depends on the application.
%When $z$ is the offer of treatment in a randomized controlled trial, for example, Assumption \ref{assump:homosked} holds automatically as a consequence of the randomization.
%Similarly, in studies based on a ``natural'' rather than a controlled experiment, one typically argues that the instrument is not merely uncorrelated with the error term but \emph{independent} of it, so that Assumption \ref{assump:homosked} follows.
\begin{thm}
  \label{pro:homosked}
  Under Assumption \ref{assump:homosked} and the conditions of Theorem \ref{pro:Lack} the difference of mis-classification rates, $(\alpha_1 - \alpha_0)$ is identified provided that $z$ takes on at least two values.
\end{thm}
\begin{proof}[Proof of Theorem \ref{pro:homosked}]
  First define
  \begin{eqnarray}
    \label{eq:mustar}
    \mu_{k\ell}^* &=&  (p_k - \alpha_0) m_{1k}^* - (p_{\ell}-\alpha_0)m_{k\ell}^* \\
    \label{eq:y2def}
    \Delta\overline{y^2} &=&  \mathbb{E}(y^2|z_k) - \mathbb{E}(y^2|z_\ell)\\
    \label{eq:yTdef}
    \Delta\overline{yT} &=&  \mathbb{E}(yT|z_k) - \mathbb{E}(yT|z_\ell)
  \end{eqnarray}
  By iterated expectations it follows, after some algebra, that
  \begin{eqnarray}
    \label{eq:y2}
    \Delta\overline{y^2} &=& \beta \mathcal{W} (p_k - p_\ell)  + 2 \mathcal{W} \mu_{k\ell}^* \\
    \label{eq:yT}
    \Delta\overline{yT} &=& (1-\alpha_1)\mathcal{W}(p_k - p_\ell) + \mu_{k\ell}^* 
  \end{eqnarray}
  Now, solving Equation \ref{eq:yT} for $\mu_{k\ell}^*$, substituting the result into Equation \ref{eq:y2} and rearranging,
  \begin{equation}
    \mathcal{R} \equiv \beta - 2(1-\alpha_1)\mathcal{W} = \frac{\Delta\overline{y^2} - 2 \mathcal{W}\Delta\overline{yT}}{\mathcal{W}(p_k - p_\ell)}.
    \label{eq:Rdef}
  \end{equation}
  Since $\mathcal{W}$ is identified it follows that $\mathcal{R}$ is identified.
  Rearranging the preceding equality and substituting $\beta=\mathcal{W}(1-\alpha_0 -\alpha_1)$ to eliminate $\beta$, we find that
  \begin{equation}
    \alpha_1 - \alpha_0 = 1 + \mathcal{R}/\mathcal{W}.
   \label{eq:aDiff}
  \end{equation}
  Because both $\mathcal{R}$ and $\mathcal{W}$ are identified, it follows that the difference of error rates is also identified.
\end{proof}

The preceding result can be used in several ways.
One possibility is to test for the presence of mis-classification error.
If the treatment is measured without error, then $\alpha_0$ must equal $\alpha_1$.
By examining the identified quantities $\mathcal{R}$ and $\mathcal{W}$, one could possibly discover that this requirement it violated.
Moreover, in some settings mis-classification may be one-sided.
In a smoking and birthweight example, it seems unlikely that mothers who did \emph{not} smoke during pregnancy would falsely claim to have smoked.
If either of $\alpha_0, \alpha_1$ is known, Theorem \ref{pro:homosked} point identifies the unknown error rate and hence $\beta$, using the fact that $\beta=\mathcal{W}(1-\alpha_0-\alpha_1)$.
When neither of the error rates is known \emph{a priori}, the same basic idea can be used to construct \emph{bounds} for $\beta$.

We now show that by augmenting Theorem \ref{pro:homosked} with information on conditional \emph{third} moments, we can point identify $\beta$.

\begin{thm}
  \label{thm:skew}
  Under Assumptions \ref{assump:homosked}-\ref{assump:skew} and the conditions of Theorem \ref{pro:Lack}, the mis-classification rates $\alpha_0$ and $\alpha_1$ and the treatment effect $\beta$ are identified provided that $z$ takes on at least two values.
\end{thm}

\begin{proof}[Proof of Theorem \ref{thm:skew}]
  First define 
  \begin{eqnarray}
    \label{eq:vstar}
  v^*_{tk} &=&  \mathbb{E}(u^2|T^*=t, z = z_k)\\
    \label{eq:lambda}
  \lambda_{k\ell}^* &=& (p_k - \alpha_0) v_{1k}^* - (p_\ell - \alpha_0) v_{1\ell}^*\\
    \label{eq:y3def}
    \Delta\overline{y^3} &=&  \mathbb{E}(y^3|z_k) - \mathbb{E}(y^3|z_\ell)\\
    \label{eq:y2Tdef}
    \Delta\overline{y^2T} &=&  \mathbb{E}(y^2T|z_k) - \mathbb{E}(y^2T|z_\ell)
  \end{eqnarray}
  where $u$, as above, is defined as $\varepsilon + c$.
  By iterated expectations it follows, after some algebra, that
  \begin{eqnarray}
    \label{eq:y3}
    \Delta\overline{y^3} &=& \beta^2 \mathcal{W} (p_k - p_\ell)  + 3 \beta \mathcal{W} \mu_{k\ell}^* + 3 \mathcal{W} \lambda^*_{k\ell}\\
    \label{eq:y2T}
    \Delta\overline{y^2T} &=& \beta(1-\alpha_1)\mathcal{W}(p_k - p_\ell) + 2(1-\alpha_1)\mathcal{W}\mu_{k\ell}^* + \lambda_{k\ell}^* 
  \end{eqnarray}
  where, as above, the identified quantity $\mathcal{W}$ equals $\beta/(1 - \alpha_0 - \alpha_1)$ and $\mu_{k\ell}^*$ is as defined in Equation \ref{eq:mustar}.
  Now, substituting for $\lambda^*_{k\ell}$ in Equation \ref{eq:y3} using Equation \ref{eq:y2T} and rearranging, we find that
  \begin{equation}
    \Delta \overline{y^3} - 3 \mathcal{W} \Delta\overline{y^2T} = \beta \mathcal{W} (p_k - p_\ell)\left[ \beta - 3 \mathcal{W} (1-\alpha_1) \right]+ 3\mathcal{W}\mathcal{R}\mu^*_{k\ell}
    \label{eq:diffy3}
  \end{equation}
  where $\mathcal{R}$ is as defined in Equation \ref{eq:Rdef}.
  Now, using Equation \ref{eq:yT} to eliminate $\mu_{k\ell}^*$ from the preceding equation, we find after some algebra that 
  \begin{equation}
    \mathcal{S} \equiv \beta^2 - 3\mathcal{W}(1-\alpha_1) (\beta + \mathcal{R}) = \frac{\Delta\overline{y^3} - 3 \mathcal{W}\left[ \Delta\overline{y^2T}+\mathcal{R}\Delta\overline{yT} \right]}{\mathcal{W}(p_k - p_\ell)}.
    \label{eq:Sdef}
  \end{equation}
  Notice that $\mathcal{S}$ is identified.
  Finally, by eliminating $\beta$ from the preceding expression using Equation \ref{eq:Rdef}, we obtain a quadratic equation in $(1-\alpha_1)$, namely 
  \begin{equation}
    2\mathcal{W}^2 (1-\alpha_1)^2 + 2 \mathcal{R}\mathcal{W} (1-\alpha_1) + (\mathcal{S} -\mathcal{R}^2) = 0.
    \label{eq:quadratic}
  \end{equation}
  Note that, since, $\mathcal{W}, \mathcal{R}$ and $\mathcal{S}$ are all identified, we can solve Equation \ref{eq:quadratic} for $(1-\alpha_1)$.
The solutions are as follows
\begin{equation}
  (1 - \alpha_1) = \frac{1}{2} \left( -\frac{\mathcal{R}}{\mathcal{W}} \pm  \frac{1}{\mathcal{W}}\sqrt{3\mathcal{R}^2 - 2 \mathcal{S}}\right)
  \label{eq:quadsolutions}
\end{equation}
It can be shown that $3\mathcal{R}^2 - 2\mathcal{S} = \left[ \mathcal{R} + 2 \mathcal{W}(1-\alpha_1) \right]^2$ so the quantity under the radical is guaranteed to be positive, yielding two real solutions.
One of these is $(1-\alpha_1)$, but what about the other root?
Using Equation \ref{eq:aDiff} we can re-express Equation \ref{eq:quadratic} as a quadratic in $\alpha_0$.
Surprisingly, after simplifying, we obtain a quadratic with \emph{identical} coefficients.
This implies that the second root of Equation \ref{eq:quadratic} identifies $\alpha_0$.
Since we know the sign of the difference $\alpha_1 - \alpha_0$ from Theorem \ref{pro:homosked}, we know which mis-classification rate is larger and hence can correctly label the two roots.
Finally, substituting into $\beta = \mathcal{W}(1-\alpha_0-\alpha_1)$, we identify the treatment effect.
\end{proof}

Note that, in contrast to all other results in the literature \citep{KRS, BBS, FL, Mahajan, Lewbel}, our proof does \emph{not} require the assumption that $\alpha_0 + \alpha_1 <1$ to identify $\beta$.


