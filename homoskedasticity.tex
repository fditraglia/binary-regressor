%!TEX root = ./main.tex
\section{Identification Based on Higher Moments}
Having shown that the moment conditions from Table \ref{tab:observables} do not identify $\beta$ regardless of the value of $K$, we now consider exploiting the information contained in higher moments of $y$. 
When $z$ is not merely mean-independent but in fact \emph{statistically} independent of $\varepsilon$, as in a randomized controlled trial or a true natural experiment, the following assumptions hold automatically.
\begin{assump}[Second Moment Independence]
  $\mathbb{E}[\varepsilon^2|z]=\mathbb{E}[\varepsilon^2]$
  \label{assump:homosked}
\end{assump}
\begin{assump}[Third Moment Independence]
  $\mathbb{E}[\varepsilon^3|z]=\mathbb{E}[\varepsilon^3]$
  \label{assump:skew}
\end{assump}
%When combined with the usual IV assumption, $\mathbb{E}[u|z]=0$, Assumption \ref{assump:homosked} implies that $Var(\varepsilon|z) = Var(\varepsilon)$.
%Whether this assumption is reasonable, naturally, depends on the application.
%When $z$ is the offer of treatment in a randomized controlled trial, for example, Assumption \ref{assump:homosked} holds automatically as a consequence of the randomization.
%Similarly, in studies based on a ``natural'' rather than a controlled experiment, one typically argues that the instrument is not merely uncorrelated with the error term but \emph{independent} of it, so that Assumption \ref{assump:homosked} follows.
\begin{thm}
  \label{pro:homosked}
  Under Assumption \ref{assump:homosked} and the conditions of Theorem \ref{pro:Lack} the difference of mis-classification rates, $(\alpha_1 - \alpha_0)$ is identified provided that $z$ takes on at least two values.
\end{thm}
\begin{proof}[Proof of Theorem \ref{pro:homosked}]
  First define
  \begin{eqnarray}
    \label{eq:mustar}
    \mu_{k\ell}^* &=&  (p_k - \alpha_0) m_{1k^*} - (p_{\ell}-\alpha_0)m_{k\ell} \\
    \label{eq:y2def}
    \Delta\overline{y^2} &=&  \mathbb{E}(y^2|z_k) - \mathbb{E}(y^2|z_\ell)\\
    \label{eq:yTdef}
    \Delta\overline{yT} &=&  \mathbb{E}(yT|z_k) - \mathbb{E}(yT|z_\ell)
  \end{eqnarray}
  By iterated expectations it follows, after some algebra, that
  \begin{eqnarray}
    \label{eq:y2}
    \Delta\overline{y^2} &=& \beta \mathcal{W} (p_k - p_\ell)  + 2 \mathcal{W} \mu_{k\ell}^* \\
    \label{eq:yT}
    \Delta\overline{yT} &=& (1-\alpha_1)\mathcal{W}(p_k - p_\ell) + \mu_{k\ell}^* 
  \end{eqnarray}
  Now, solving Equation \ref{eq:yT} for $\mu_{k\ell}^*$ and substituting the result into Equation \ref{eq:y2} and rearranging,
  \begin{equation}
    \mathcal{R} \equiv \beta - 2(1-\alpha_1)\mathcal{W} = \frac{\Delta\overline{y^2} - 2 \mathcal{W}\Delta\overline{yT}}{\mathcal{W}(p_k - p_\ell)}.
    \label{eq:Rdef}
  \end{equation}
  Since $\mathcal{W}$ is identified it follows that $\mathcal{R}$ is identified.
  Rearranging the preceding equality and substituting $\beta=\mathcal{W}(1-\alpha_0 -\alpha_1)$ to eliminate $\beta$, we find that
  \begin{equation}
    \alpha_1 - \alpha_0 = 1 + \mathcal{R}/\mathcal{W}.
   \label{eq:aDiff}
  \end{equation}
  Because both $\mathcal{R}$ and $\mathcal{W}$ are identified, it follows that the difference of error rates is also identified.
\end{proof}

\begin{thm}
  \label{thm:skew}
  Under Assumptions \ref{assump:homosked}-\ref{assump:skew} and the conditions of Theorem \ref{pro:Lack}, the mis-classification rates $\alpha_0$ and $\alpha_1$ and the treatment effect $\beta$ are identified provided that $z$ takes on at least two values.
\end{thm}

\begin{proof}[Proof of Theorem \ref{thm:skew}]
  From the proof of Theorem \ref{pro:Lack}, $\mathcal{W} = \beta/(1-\alpha_0 - \alpha_1)$ is identified. 
  And from the proof of \ref{pro:homosked}, $(\alpha_1 - \alpha_0) = 1 + \mathcal{R}/\mathcal{W}$ is identified.
  Thus, it suffices to identify $(1-\alpha_1)$. 
\end{proof}
