%!TEX root = ./main.tex
\section{Identification Based on Higher Moments}
Having shown that the moment conditions from Table \ref{tab:observables} do not identify $\beta$ regardless of the value of $K$, we now consider exploiting the information contained in higher moments of $y$. 
When $z$ is not merely mean-independent but in fact \emph{statistically} independent of $\varepsilon$, as in a randomized controlled trial or a true natural experiment, the following assumptions hold automatically.
\begin{assump}[Second Moment Independence]
  $\mathbb{E}[\varepsilon^2|z]=\mathbb{E}[\varepsilon^2]$
  \label{assump:homosked}
\end{assump}
\begin{assump}[Third Moment Independence]
  $\mathbb{E}[\varepsilon^3|z]=\mathbb{E}[\varepsilon^3]$
  \label{assump:skew}
\end{assump}
%When combined with the usual IV assumption, $\mathbb{E}[u|z]=0$, Assumption \ref{assump:homosked} implies that $Var(\varepsilon|z) = Var(\varepsilon)$.
%Whether this assumption is reasonable, naturally, depends on the application.
%When $z$ is the offer of treatment in a randomized controlled trial, for example, Assumption \ref{assump:homosked} holds automatically as a consequence of the randomization.
%Similarly, in studies based on a ``natural'' rather than a controlled experiment, one typically argues that the instrument is not merely uncorrelated with the error term but \emph{independent} of it, so that Assumption \ref{assump:homosked} follows.
\begin{thm}
  \label{pro:homosked}
  Under Assumption \ref{assump:homosked} and the conditions of Theorem \ref{pro:Lack} the difference of mis-classification rates, $(\alpha_1 - \alpha_0)$ is identified. 
\end{thm}
\begin{proof}[Proof of Theorem \ref{pro:homosked}]
  Let $s^2_k$ denote the conditional variance of $y$ given that $z=z_k$.
  Then we have:
\begin{equation}
  s_k^2 = \beta^2 \sigma^2_{T^*|z_k} + \sigma^2_\varepsilon + 2\beta \sigma_{T^*,\varepsilon|z_k}
\end{equation}
where $\sigma^2_{T^*|z_k}$ denotes the conditional variance of $T^*$ given that $z = z_k$, $\sigma_{T^*,\varepsilon|z_k}$ denotes the conditional covariance between $T^*$ and $\varepsilon$ given that $z=z_k$ and we have used the fact that $Var(\varepsilon|z)$ does not depend on $z$ by Assumption \ref{assump:homosked}.
Taking differences across two values $z_k$ and $z_\ell$ of the instrument, we have
\begin{equation}
  s^2_k - s^2_\ell = \beta^2\left( \sigma^2_{T^*|z_k} - \sigma^2_{T^*|z_\ell} \right) + 2\beta \left( \sigma_{T^*,\varepsilon|z_k} - \sigma_{T^*,\varepsilon|z_\ell} \right).
\end{equation}
Now, by the definition of $p^*_k$ and Equation \ref{eq:pkstar},
\begin{equation}
  \sigma^2_{T^*|z_k}= p^*_k (1-p^*_k) = \frac{(p_k - \alpha_0)(1 - p_k - \alpha_1)}{(1 - \alpha_0 - \alpha_1)^2} 
\end{equation}
which implies that 
\begin{equation}
  \left( \sigma^2_{T^*|z_k} - \sigma^2_{T^*|z_\ell} \right)= \frac{p_k(1-p_k) - p_\ell(1-p_\ell) + (\alpha_0 - \alpha_1)(p_k - p_\ell)}{(1 - \alpha_0 - \alpha_1)^2}.
\end{equation}
Turning our attention to the conditional covariances, note that
\begin{equation}
  \sigma_{T^*,\varepsilon|z_k} = E_{T^*|z}\left[E\left( T^*\varepsilon|T^*z_k \right)  \right] = p_k^* (m^*_{1k}-c) 
\end{equation}
since $E[\varepsilon|z] = 0$.
Thus, by Equation \ref{eq:pkstar},
\begin{equation}
  \left( \sigma_{T^*,\varepsilon|z_k} - \sigma_{T^*,\varepsilon|z_\ell} \right) = \frac{(p_k - \alpha_0)(m^*_{1k}-c) - (p_\ell - \alpha_0)(m^*_{1\ell}-c)}{1 - \alpha_0 - \alpha_1} 
\end{equation}
Now, substituting $\mathcal{W} = \beta/(1-\alpha_0-\alpha_1)$, we find that
\begin{align*}
  s^2_k - s^2_\ell = \mathcal{W}^2\left[p_k(1-p_k) - p_\ell(1-p_\ell) + (\alpha_0 - \alpha_1)(p_k - p_\ell)\right]\\  
  + 2\mathcal{W}\left[(p_k - \alpha_0)(m^*_{1k}-c) - (p_\ell - \alpha_0)(m^*_{1\ell}-c)\right]
\end{align*}
Now, define the observable
\begin{equation*}
  \widetilde{\mathcal{W}}_{k\ell} = \frac{\mathbb{E}[yT|z_k] - \mathbb{E}[yT|z_\ell]}{p_k - p_\ell}
\end{equation*}
Some algebra shows that, under our assumptions,
\begin{align*}
(p_k - \alpha_0)&(m^*_{1k}-c) - (p_\ell - \alpha_0)(m^*_{1\ell}-c) =\\
& (p_k - p_\ell)\left[ \widetilde{W}_{k\ell} - \mathbb{E}[y] - \mathcal{W}\left\{ (1 - p) + (\alpha_0 - \alpha_1) \right\} \right]
\end{align*}
where $p = E[T]$.
Substituting this into our expression for $s_k^2 - s_\ell^2$ allows us to solve for $\alpha_0 - \alpha_1$ in terms of observables, specifically:
\begin{equation*}
  \alpha_0 - \alpha_1 = (2p - 1 - p_k - p_\ell) + \frac{2(\widetilde{W}_{k\ell} - \mathbb{E}[y])}{\mathcal{W}} - \frac{s_k^2 - s_\ell^2}{(p_k - p_\ell)\mathcal{W}^2}.
\end{equation*}
\end{proof}

\begin{thm}
  \label{thm:skew}
  Under Assumptions \ref{assump:homosked}-\ref{assump:skew} and the conditions of Theorem \ref{pro:Lack}, the mis-classification rates $\alpha_0$ and $\alpha_1$ and the treatment effect $\beta$ are identified.
\end{thm}

\begin{proof}[Proof of Theorem \ref{thm:skew}]
  From the proof of Theorem \ref{pro:Lack}, $\mathcal{W} = \beta/(1-\alpha_0 - \alpha_1)$ is identified. 
  And from the proof of \ref{pro:homosked}, $(\alpha_1 - \alpha_0) = 1 + \mathcal{R}/\mathcal{W}$ is identified.
  Thus, it suffices to identify $(1-\alpha_1)$. 
\end{proof}
