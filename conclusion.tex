%!TEX root = ./main.tex
\section{Conclusion}
This paper has presented the first point identification result for the effect of an endogenous, binary, mis-measured treatment using a discrete instrument.
While our results require us to impose stronger conditions on the instrument, these conditions are satisfied in a number of empirically relevant examples, for example randomized controlled trials and true natural experiments.
We obtain identification by augmenting conditional first moments with additional information contained in second and third moments and further derive a partial identification result based on first and second moments alone.
In addition, and contrary to an incorrect previous result in \cite{Mahajan}, we showed that appealing to higher moments is necessary if one wishes to obtain identification: first moment information alone cannot identify the causal effect of an endogenous, mis-classified binary treatment regardless of the number of values the instrument may take.

This project is still in progress.
The present draft has focused on establishing identification in the simplest possible way: by finding explicit solutions to equations that arise from imposing particular conditional moment assumptions on $\varepsilon$.
In practice, however, it may be more natural to think of these as implications of stronger \emph{independence} assumptions that arise from the specific example under consideration.
Work currently in progress explores the consequences of imposing independence directly.
For example, the traditional non-differential measurement error assumption that $\varepsilon$ is conditionally independent of $T$ given $T^*,z, \mathbf{x}$ indeed implies our Equation \ref{eq:nondiff}.
Our work thus far, however, indicates that under independence one can derive sharp bounds on $\alpha_0$ and $\alpha_1$ without making \emph{any} assumption regarding the relationship between $z$ and $\varepsilon$.
Moreover, preliminary results suggest that the resulting bounds can be quite tight in practice.
Results along these lines should yield a simple robustness test allowing researchers who are willing to assume only the standard IV assumption, rather than the stronger conditions we impose above, to explore the possibility that mis-classification has inflated their IV estimates. 
Likewise, if one is willing to assume that $z$ is in fact independent of $\varepsilon$, which indeed implies our Assumptions \ref{assump:homosked} and \ref{assump:skew}, much more information becomes available for use in estimation.
Efficiently exploiting the resulting \emph{continuum} of moment conditions should lead to substantially improved estimation of the model parameters in addition to providing a convenient test of model specification.

