%!TEX root = ./main.tex
\section{Conclusion}
This paper has presented the first point identification result for the effect of an endogenous, binary, mis-measured treatment using a discrete instrument.
While our results require us to impose stronger conditions on the instrument, these conditions are satisfied in a number of empirically relevant examples, for example randomized controlled trials and true natural experiments.
We obtain identification by augmenting conditional first moments with additional information contained in second and third moments and further derive a partial identification result based on first and second moments alone.
In addition, and contrary to an incorrect previous result in \cite{Mahajan}, we showed that appealing to higher moments is necessary if one wishes to obtain identification: first moment information alone cannot identify the causal effect of an endogenous, mis-classified binary treatment regardless of the number of values the instrument may take.
We have focused here on establishing identification and partial identification results using a particular set of moment conditions. More generally one could consider the use of additional moment conditions based on an independence assumption for the instrument.
From the standpoint of estimation, rather than identification, the use of such moments could provide efficiency gains.
Another possible avenue for future research would be to extend our framework to the fuzzy regression discontinuity setting.
