%!TEX root = ./main.tex
\section{Conclusion}
\label{sec:conclusion}

This paper has studied identification and inference for a mis-classified, binary, endogenous regressor in an additively separable model using a discrete instrumental variable.
We have shown that the only existing identification result for this model is incorrect, and gone on to derive the sharp identified set under standard first-moment assumptions from the literature.
Strengthening these assumptions to hold for second and third moments, we have established point identification for the effect of interest.
Inference in models with mis-classification error is complicated by problems of weak identification and parameters on the boundary. 
To address these challenges, we have proposed a Bonferroni-based procedure for identification robust inference, using both the moment equalities from our identification results and moment inequalities from our partial identification results.
This procedure is computationally attractive and performs well in simulations.
An interesting extension of the results presented here would be to explore the more general case of a discrete endogenous regressor subject to mis-classification error, possibly by combining our approach with the matrix factorization techniques from \cite{hu2008}.
Another interesting extension, inspired by our hybrid confidence interval heuristic from Section \ref{sec:simulation}, would be to study the transition between robust and standard inference in moment condition models.
It may be possible, for example, to adapt the techniques of \cite{andrews2016valid} in this direction to provide similar theoretical guarantees.


