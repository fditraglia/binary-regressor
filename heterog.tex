%!TEX root = ./main.tex
\section{Unobserved Heterogeneity}
\label{sec:het}
\todo[inline]{Although our results allow arbitrary observed heterogeneity, additive separability places restrictions on unobserved heterogeneity.
  Although it is not the main focus of our paper, unlike Ura, briefly comment on how these results can be interpreted, say, in a LATE context. First, the bounds stuff all goes through (???) provided one is willing to make the LATE assumptions. Higher moment restrictions do impose restrictions. Can say what happens with the second moment assumption since we already derived this: it's a restriction on the variance of the potential outcome distributions, conditional on $\mathbf{x}$.}


\subsubsection{Derivations from the Notes}

\paragraph{Is there a LATE interpretation of our results?}
Let $J \in \left\{ a, c, d, n \right\}$ index an individual's \emph{type}: always-taker, complier, defier, or never-taker.
Let $\pi_a, \pi_c, \pi_d, \pi_n$ denote the population proportions of always-takers, compliers, defiers, and never-takers.
The unconfounded type assumption is $P(J=j|z=1) = P(J=j|z=0)$.
Combined with the law of total probability, this gives
\begin{align*}
  p^*_1 &= P(T^*=1|z=1) = \pi_a + \pi_c \\
  1 - p^*_1 &= P(T^*=0|z=1) = \pi_d + \pi_n \\
  p^*_0 &= P(T^*=1|z=0) = \pi_d + \pi_a \\
  1-p^*_0 &= P(T^*=0|z=0) = \pi_n + \pi_c 
\end{align*}
Imposing no-defiers, $\pi_d = 0$, these expressions simplify to
\begin{align*}
  p^*_1 &=  \pi_a + \pi_c \\
  1 - p^*_1 &=  \pi_n \\
  p^*_0 &=  \pi_a \\
  1-p^*_0 &=  \pi_n + \pi_c 
\end{align*}
Solving for $\pi_c$, we see that
\begin{align*}
  \pi_c &= p_1^* - p_0^*\\
  \pi_a &= p_0^*\\
  \pi_n &= 1 - p_1^*
\end{align*}

Now, let $Y(1)$ indicate the potential outcome when $T^*=1$ and $Y(0)$ indicate the potential outcome when $T^*=0$.
The standard LATE assumptions (no defiers, mean exclusion, unconfounded type) imply
\begin{align*}
  \mathbb{E}\left( Y| T^* = 1, z = 1 \right) &= \left( \frac{p_0^*}{p_1^*} \right) \mathbb{E}\left[ Y(1)|J=a \right] + \left( \frac{p_1^* - p_0^*}{p_1^*} \right)\mathbb{E}\left[ Y(1)|J=c \right] \\
  \mathbb{E}\left( Y| T^* = 0, z = 0 \right) &= \left( \frac{p_1^* - p_0^*}{1 - p_0^*} \right)\mathbb{E}\left[ Y(0)|J=c \right] + \left( \frac{1 - p_1^*}{1 - p_0^*} \right)\mathbb{E}\left[ Y(0)|J=n \right]\\
  \mathbb{E}\left( Y| T^* = 1, z = 0 \right) &= \mathbb{E}\left[ Y(1)|J=a \right]\\
  \mathbb{E}\left( Y| T^* = 0, z = 1 \right) &= \mathbb{E}\left[ Y(0)|J=n \right]
\end{align*}



\paragraph{LATE Version of Theorem 2 from original Draft}
\begin{align*}
  \Delta\overline{yT} &= \mathbb{E}\left( yT|z=1 \right) - \mathbb{E}\left( yT|z=0 \right) \\
  &= (1 - \alpha_1) \left[ p_1^* \mathbb{E}\left( y|T^*=1, z=1 \right) - p_0^* \mathbb{E}\left(y|T^*=1, z=0\right) \right] \\
  & \; \; \quad \quad + \alpha_0 \left[ (1 - p_1^*)\mathbb{E}\left( y|T^*=0, z=1\right) - (1 - p_0^*)\mathbb{E}\left(y|T^*,z=0 \right) \right]
\end{align*}
So we find that
\begin{align*}
  \Delta\overline{yT} &= (p_1^* - p_0^*)\left\{ (1 - \alpha_1) \mathbb{E}\left[ Y(1)|J=c \right] - \alpha_0\mathbb{E}\left[ Y(0)|J=c \right] \right\}\\
  &= (1 - \alpha_1) \left\{ \frac{\mathbb{E}\left[ Y(1) - Y(0)|J=c \right]}{1 - \alpha_0 - \alpha_1} (p_1 - p_0) \right\} + (p_1  - p_0) \mathbb{E}\left[ Y(0)|J=c \right]
\end{align*}
Recall that the analogous expression in the homogeneous treatment effect case is
\begin{align*}
  \Delta\overline{yT} &= (1 - \alpha_1) \mathcal{W} (p_1 - p_0) + \mu_{10}^*\\
  &= (1 - \alpha_1) \left(\frac{\beta}{1 - \alpha_0 - \alpha_1}\right) (p_1 - p_0) + (p_1 - \alpha_0)m_{11}^* - (p_0 - \alpha_0)m_{10}^*
\end{align*}
while the expression for the difference of variances is 
\begin{align*}
  \Delta\overline{y^2} &= \beta \mathcal{W}(p_1 - p_0) + 2\mathcal{W} \mu_{10}^*
\end{align*}
From above we see that the analogue of $\mu_{10}^*$ in the heterogeneous treatment effects setting is $(p_1 - p_0)E\left[ Y(0)|J=c \right]$ and since the LATE is $\mathbb{E}\left[ Y(1) - Y(0) |J=c\right]$, the analogue of $\mathcal{W}$ is
\[
  \frac{\mathbb{E}\left[ Y(1) - Y(0)|J=c \right]}{1 - \alpha_0 - \alpha_1}
\]
so \emph{if} we could establish that 
\[
  \Delta\overline{y^2} =  \left( \frac{p_1 - p_0}{1 - \alpha_0 - \alpha_1} \right)\mathbb{E}\left[ Y(1) - Y(0)|J=c \right]\cdot \mathbb{E}\left[ Y(1) + Y(0) |J=c \right]
\]
in the heterogeneous treatment effects case, the proof of Theorem 2 would go through immediately.
Now, if we assume an exclusion restriction on the \emph{second} moment of $y$ an argument almost identical to the standard LATE derivation gives
\[
  \Delta\overline{y^2} = \frac{\mathbb{E}\left[ Y^2(1) - Y^2(0) |J=c \right]}{p_1^* - p_0^*} = \left( \frac{p_1 - p_0}{1 - \alpha_0 - \alpha_1} \right)\mathbb{E}\left[ Y^2(1) - Y^2(0) |J=c \right] 
\]
so we see that the necessary and sufficient condition for our proof to go through is 
\[
  \mathbb{E}\left[ Y^2(1) - Y^2(0)|J=c \right] = \mathbb{E}\left[ Y(1) - Y(0)|J=c \right]\cdot \mathbb{E}\left[ Y(1) + Y(0)|J=c \right]
\]
Rearranging, this in turn is equivalent to
\[
  \mbox{Var}\left[ Y(1)|J=c \right] = \mbox{Var}\left[ Y(0)|J=c \right]
\]

