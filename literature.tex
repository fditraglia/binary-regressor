%!TEX root = ./main.tex
\section{Related Literature}
Many treatments of interest in applied work are binary.
To take a particularly prominent example, consider treatment status in a randomized controlled trial.
Even if the randomization is pristine, which yields a valid binary instrument (the offer of treatment), subjects may select into treatment based on unobservables, and given the many real-world complications that arise in the field, measurement error may be an important concern.
As is well known, instrumental variables (IV) based on a single valid instrument suffices to recover the treatment effect in a linear model with a single endogenous regressor subject to classical measurement errors.
As is less well known, classical measurement error is in fact impossible when the regressor of interest is binary: because a true 1 can only be mis-measured as a 0 and a true 0 can only be mis-measured as a 1, the measurement error must be \emph{negatively} correlated with the true treatment status \citep{Aigner,Bollinger}. 

Measurement error in a binary regressor is usually called \emph{mis-classification}.
The simplest form of mis-classification is so-called \emph{non-differential} measurement error.
In this case, conditional on true treatment status, and possibly a set of exogenous covariates, the measurement error is assumed to be uncorrelated with all other variables in the system.
Even under this comparatively mild departure from classical measurement error, the IV estimator is inconsistent \citep{BBS, KRS}. 
Moreover, the probability limit of the IV estimator does not depend on whether the treatment is endogenous or not \citep{FL}.

When the treatment is in fact \emph{exogenous}, however, a valid instrument suffices to recover the treatment effect using a non-linear GMM estimator.
\cite{BBS} and \cite{KRS} more-or-less simultaneously pointed this out in a setting in which \emph{two} alternative measures of treatment are available, both subject to non-differential measurement error.
In essence, one measure serves as an instrument for the other although the estimator is quite different from IV.\footnote{Ignoring covariates, the observable moments in this case are the joint probability distribution of the two binary treatment measures and the conditional means of the outcome variable given the two measures. Although the system is highly non-linear, it can be manipulated to yield an explicit solution for the treatment effect provided that the true treatment is exogenous.}
Subsequently, \cite{FL} correctly note that an instrumental variable can take the place of one of the measures of treatment in a linear model with an exogenous treatment, allowing one to implement a variant of the GMM estimator proposed by \cite{BBS} and \cite{KRS}.
However, as we will show below, the assumptions required to obtain this result are are stronger than \cite{FL} appear to realize: the usual IV assumption that the instrument is mean independent of the regression error is insufficient for identification. 
\cite{Mahajan} extends the results of \cite{BBS} and \cite{KRS} to a more general nonparametric regression setting using a binary instrument in place of one of the treatment measures. 
Although unaware of \cite{FL}, \cite{Mahajan} makes the correct assumption over the instrument and treatment to guarantee identification of the conditional mean function.
When the treatment is in fact exogenous, this coincides with the treatment effect.
\cite{Lewbel} provides a related identification result in the same model as \cite{Mahajan} under different assumptions.
In particular, the variable that plays the role of the ``instrument'' need not satisfy the exclusion restriction provided that it does not interact with the treatment and takes on at least three distinct values. 

% Eventually, need to talk about Hausman, and Tanguay Brachet here!

% Need to figure out how Card relates to these as well.  It looks like he does not in fact use two measures to estimate the effect of union status on wages. Instead he uses a two-period panel dataset and examines external information comparing employer and employee reporting of union status. This leads him to propose the assumption that the ``up'' and ``down'' mis-classification probabilities are equal, since it fits this external dataset well. This is the ``quasi-classical'' measurement error case that we talked about previously. There is only one measurement error parameter and presumably the panel dataset allows him to identify it.

Much less is known about the case in which the treatment, in addition to suffering from non-differential measurement error, is also endogenous.
Only two papers consider this case.
\cite{FL} briefly discuss the prospects for identification in this setting.
Although they do not provide a formal proof they argue, in the context of their parametric linear model, that the treatment effect is unlikely to be identified unless one is willing to impose strong and somewhat unnatural conditions. 
The second paper that considers this case is \cite{Mahajan}.
He extends his main result to the case of an endogenous treatment, providing an explicit proof of identification under the usual IV assumption in a model with additively separable errors.
Although their discussion does not apply to the non-parametric case, \citeauthor{FL}'s intuition turns out to be right: \citeauthor{Mahajan}'s proof is incorrect, as we prove below using a convenient notational framework introduced in the following section.

\subsection{New papers to cite\dots}

\cite{Card} need to decide how/if to cite this paper.

\cite{SongSchennachWhite} continuous treatment, supposedly endog but actually exog conditional on covariates, also use repeated measure of treatment.

\cite{song2015} General semi-parametric moment condition models but requires continuous regressor and IV. 

\cite{HuShiuWoutersen} require continuous instrument, continuous regressor. Identify ratio of partial effects.

Survey articles on nonlinear measurement error models: \cite{chensurvey} and \cite{SchennachSurvey}.

\cite{lewbel1997} continuous, exogenous treatment, no instrument, use higher moments to construct instruments.

\cite{lewbel2012} Regression with continuous, possibly endogenous regressor subject to classical measurement error. Get identification by assuming heteroskedastic errors.
Also partial identification under weaker assumption.

\cite{ChenHuLewbel} binary treatment, no endogeneity, no instrument, higher moments.

\cite{ChenHuLewbel2} Non-parametric regression model with mis-measured exogenous binary regressor. No instrument, get non-parametric identification from higher moments of the error distribution.

\cite{shiu2015} sample selection model in which outcomes are missing not-at-random, requires instrument with known parametric first stage and an additional excluded regressor.

\cite{Hausman} regression model with mis-classified discrete \emph{outcome}. Recover regression parameters and mis-classification rates using parametric model (logit/probit). Also suggest semi-parametric procedure.
In principle, this could be used as a first stage in an IV setup but two problems.
First, probably don't want to make the parametric assumption.
Second, if you try to do without it, really have identification at infinity so the instrument needs to be continuous and have large support: exclusively using information in the tails to estimate the mis-classification probabilities.

\cite{molinari} looks at identifying a discrete probability distribution from mis-classified data.

\cite{hu2008} seems to essentially re-do Mahajan with a $k$-valued treatment.

\cite{Bollinger} partial identification, regression with exogenous, mis-measured, binary regressor

\cite{BollingerHasseltWP} partial identification, regression with mis-measured, exogenous binary treatment

\cite{HasseltBollinger} exogenous treatment, no instrument.


\cite{kreider2007} differences in employment status: disabled versus not. Disability is mis-measured, nasty measurement error. Partial identification, measurement error may be non-differential, binary outcome but what is it about????

\cite{kreider2012} effects of food stamps on health outcomes of children, participation in SNAP (food stamp program) is both mis-reported an endogenous.
Don't require non-differential measurement error.
Partial identification results relying on auxiliary data.

\cite{das2005} non-parametric IV model, discrete endogenous regressors. Doesn't look at measurement error.

\cite{ChenHongTamer} moment condition model, mis-measured data (either regressors or outcomes), allow any amount of correlation between true vars and measurement error. Use an auxiliary dataset in which both truth and mis-measured are observed and need conditional distribution of truth given measurement to be the same in the auxiliary sample and the real sample. 

\cite{schennach2004} nonlinear errors-in-variables model, continuous treatment, no endogeneity, no instrument, continuous treatment. Uses repeated measure for each of the mis-measured variables. 
\cite{schennach2007} does the same thing but with instrument rather than repeated measure.
\cite{HuShennach} same as prev but with non-classical measurement error and an instrument. Can't directly extend the Lewbel/Mahajan stuff to the case of continuous regressor, use an assumption that some ``measure of location'' of the measurement error distribution. 

