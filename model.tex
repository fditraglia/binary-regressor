%!TEX root = ./main.tex
\section{The Model}
Let $T^*$ be a binary indicator of true treatment status, possibly endogenous, $\mathbf{x}$ be a vector of exogenous covariates, and $y$ be an outcome of interest where
\begin{equation}
  y = h(T^*, \mathbf{x}) + \varepsilon
  \label{eq:model}
\end{equation}
and $\varepsilon$ is mean zero.
Since $T^*$ is potentially endogenous, $\mathbb{E}[\varepsilon|T^*,\mathbf{x}]$ may not be zero.
Now let $z$ be a discrete instrumental variable with support set $\{z_k\}_{k=1}^K$ satisfying the usual instrumental variables assumption, namely $\mathbb{E}[\varepsilon|z,\mathbf{x}]=0$.
We assume throughout that $z$ is a relevant instrument for $T^*$, in other words 
\begin{equation}
\mathbb{P}(T^*=1|\mathbf{x},z_j)\neq \mathbb{P}(T^*=1|\mathbf{x},z_k), \quad \forall k\neq j.
\label{eq:relevance}
\end{equation}
Our goal is to estimate the average treatment effect (ATE) function
\begin{equation}
  \tau(\mathbf{x}) = h(1,\mathbf{x}) - h(0,\mathbf{x}).
  \label{eq:ATE}
\end{equation}
We maintain throughout that $\tau(\mathbf{x}) \neq 0$.
If it were, this would imply that $T^*$ is irrelevant for $y$ which can be directly tested regardless of whether any mis-classification is present and regardless of whether $T^*$ is endogenous.\footnote{This is because, as we will see below, the Wald Estimator is identified and is proportional to the treatment effect. This estimator exists provided that we have a valid and relevant instrument that takes on at least two values.} 

Now, suppose we observe not $T^*$ but a noisy measure $T$ polluted by non-differential measurement error.
In particular, we assume that
\begin{eqnarray}
  \label{eq:a0}
  \mathbb{P}(T = 1| T^* = 0, \mathbf{x}, z)  &=&  \alpha_0(\mathbf{x})\\
  \label{eq:a1}
  \mathbb{P}(T = 0| T^* = 1, \mathbf{x}, z)  &=&  \alpha_1(\mathbf{x})
\end{eqnarray}
and additionally that
\begin{equation}
  \mathbb{E}[\varepsilon|T^*,T,\mathbf{x},z] =  \mathbb{E}[\varepsilon|T^*,\mathbf{x},z]
  \label{eq:nondiff}
\end{equation}
Equations \ref{eq:a0}--\ref{eq:a1} amount to the assumption that $z$ and $T$ are conditionally independent given $(T^*,\mathbf{x})$. 
In other words, $z$ provides no additional information about the process that causes $T$ to be mis-classified above that already contained in $T^*$ and $\mathbf{x}$.
In contrast, we allow for the possibility that measurement error process \emph{does} depend on the exogenous covariates $\mathbf{x}$. 
Equation \ref{eq:nondiff} states that, given knowledge of true treatment status, the instrument and the exogenous covariates, the \emph{observed} treatment status contains no information about the mean of the regression error term.  
The assumptions on the measurement error process contained in Equations \ref{eq:a0}--\ref{eq:nondiff} are standard in the literature.
Another standard assumption is the condition
\begin{equation}
  \alpha_0(\mathbf{x}) + \alpha_1(\mathbf{x}) < 1
  \label{eq:alphaIneq}
\end{equation}
which rules out the possibility that $1-T$ is a better measure of $T^*$ than $T$ is, and vice-versa.
While we do not rely on this condition below, it appears, for example in \cite{Mahajan}, and \cite{Lewbel}, among others.

Our arguments below, like those of \cite{Mahajan} and \cite{Lewbel}, proceed by holding the exogenous covariates \emph{fixed} at some level $\mathbf{x}_a$.
As such, there is no loss of generality from suppressing dependence on $\mathbf{x}$ in our notation. 
It should be understood throughout that any conditioning statements are evaluated at $\mathbf{x}=\mathbf{x}_a$.
To this end let $c = h(0,\mathbf{x}_a)$ and define $\beta = h(1,\mathbf{x}_a) - h(0,\mathbf{x}_a)$.
Using this notation, Equation \ref{eq:model} can be re-expressed as a simple linear model, namely 
\begin{equation}
  y = \beta T^* + u 
  \label{eq:linear}
\end{equation}
where we define $u = c + \varepsilon$, an error term that need not be mean zero.
In the context of Equation \ref{eq:linear} the only observable information consists of the moments of $y$, conditional on $T,z$, the conditional probabilities of $T$ given $z$, and the marginal probabilities of $z$.
For the moment, following the existing literature, we will restrict attention to the conditional mean of $y$.
Let $\bar{y}_{t,k}$ denote $\mathbb{E}[y|T=t,z=z_k]$, let $p_k$ denote $\mathbb{P}(T=1|z=z_k)$ and let $q_k = \mathbb{P}(z=z_k)$.
Table \ref{tab:observables} depicts the observable moments for this problem.

\begin{table}
  \centering
  \begin{tabular}{c|c|c|c|c|}
    \multicolumn{1}{c}{}& \multicolumn{1}{c}{$z=1$} &\multicolumn{1}{c}{$z=1$} & \multicolumn{1}{c}{\dots} &\multicolumn{1}{c}{$z=K$}\\
    \cline{2-5}
    $T=0$ & \diagbox[dir=NE]{$\bar{y}_{01}$}{$p_{01}$} & \diagbox[dir=NE]{$\bar{y}_{02}$}{$p_{02}$} & \dots &\diagbox[dir=NE]{$\bar{y}_{0K}$}{$p_{0K}$}\\
    \cline{2-5}
    $T=1$ & \diagbox[dir=NE]{$\bar{y}_{11}$}{$p_{11}$} & \diagbox[dir=NE]{$\bar{y}_{12}$}{$p_{12}$} & \dots &\diagbox[dir=NE]{$\bar{y}_{1K}$}{$p_{1K}$}\\
    \cline{2-5}
  \end{tabular}
  \caption{Observables, using the shorthand $p_{0k}=q_k(1-p_k)$ and $p_{1k}=q_kp_k$.}
  \label{tab:observables}
\end{table}

The observed cell means $\bar{y}_{tk}$ depend on a number of unobservable parameters which we now define. 
Let $m^*_{tk}$ denote the conditional mean of $u$ given $T^*=t$ and $z=z_k$, $\mathbb{E}[u|T^*=t, z=z_k]$, and let $p^*_k$ denote $\mathbb{P}(T^*=1|z=z_k)$.
These quantities are depicted in Table \ref{tab:unobservables}.
By the Law of Total Probability and the definitions of $p_k$ and $p_k^*$,
\begin{eqnarray*}
  p_k &=&  \mathbb{P}(T=1|z=z_k,T^*=0)(1-p^*_{k}) + \mathbb{P}(T=1|z=z_k,T^*=1)p^*_k \\
  &=& \alpha_0 (1-p^*_k) + (1 - \alpha_1) p_k^* 
\end{eqnarray*}
since the misclassification probabilities do not depend on $z$ by Equations \ref{eq:a0}--\ref{eq:a1}.
Rearranging, 
\begin{equation}
  p_k^* = \frac{p_k - \alpha_0}{1 - \alpha_0 - \alpha_1}, \quad
  1 -p_k^* = \frac{1 - p_k - \alpha_1}{1 - \alpha_0 - \alpha_1}. 
  \label{eq:pkstar}
\end{equation}
Equation \ref{eq:pkstar} implies that $p_k^*$ is observable up to knowledge of the mis-classification rates $\alpha_0,\alpha_1$ since $p_k$ is observable.
Thus, the full set of parameters needed to characterize the model in Equation \ref{eq:linear} consists of $\beta, \alpha_0, \alpha_1$ and the conditional means of $u$, namely $m^*_{tk}$ for a total of $2K+3$ parameters.
In contrast, there are only $2K$ available moment conditions, namely:
\begin{align}
  \label{eq:MC0}
  \hat{y}_{0k} &=\frac{\alpha_1(p_k - \alpha_0)(\beta + m_{1k}^*) + (1 - \alpha_0)(1 - p _k -  \alpha_1)m_{0k}^*}{1 - \alpha_0 - \alpha_1} \\[1.5ex]
  \label{eq:MC1}
  \hat{y}_{1k} &= \frac{(1-\alpha_1)(p_k - \alpha_0)(\beta+m_{1k}^*)+  \alpha_0(1-p_k - \alpha_1)m_{0k}^*}{1-\alpha_0 - \alpha_1}
\end{align}
by the Law of Iterated Expectations, where the observables on the left hand side are defined according to $\hat{y}_{0k} = (1-p_k)\bar{y}_{0k}$ and $\hat{y}_{1k}= p_k \bar{y}_{1k}$.
Notice that the observable ``weighted'' cell mean $\hat{y}_{tk}$ depends on both $m^*_{tk}$ \emph{and} $m^*_{1-t,k}$ since the cell in which $T=t$ from Table \ref{tab:observables} is in fact a mixture of both the cells $T^*=0$ and $T^*=1$ from Table \ref{tab:unobservables}, for a particular column $k$.

\begin{table}
  \centering
  \begin{tabular}{c|c|c|c|c|}
    \multicolumn{1}{c}{}& \multicolumn{1}{c}{$z=1$} &\multicolumn{1}{c}{$z=1$} & \multicolumn{1}{c}{\dots} &\multicolumn{1}{c}{$z=K$}\\
    \cline{2-5}
    $T^*=0$ & \diagbox[dir=NE]{$m^*_{01}$}{$p^*_{01}$} & \diagbox[dir=NE]{$m^*_{02}$}{$p^*_{02}$} & \dots &\diagbox[dir=NE]{$m^*_{0K}$}{$p^*_{0K}$}\\
    \cline{2-5}
    $T^*=1$ & \diagbox[dir=NE]{$m^*_{11}$}{$p^*_{11}$} & \diagbox[dir=NE]{$m^*_{12}$}{$p^*_{12}$} & \dots &\diagbox[dir=NE]{$m^*_{1K}$}{$p^*_{1K}$}\\
    \cline{2-5}
  \end{tabular}
  \caption{Unobservables, using the shorthand $p^*_{0k}=q_k(1-p^*_k)$ and $p^*_{1k}=q_kp_k^*$.}
  \label{tab:unobservables}
\end{table}

Clearly we have fewer equations than unknowns.
What additional restrictions could we consider imposing on the system? 
In a very interesting paper, \cite{Lewbel} proposes using a three-valued ``instrument'' that does \emph{not} satisfy the exclusion restriction.
By assuming instead that there is no \emph{interaction} between the instrument and the treatment, he is able to prove identification of the treatment effect.
Using our notation it is very easy to see why and how \citeauthor{Lewbel}'s argument works.
His moment conditions are equivalent to Equations \ref{eq:MC0} and \ref{eq:MC1} with the additional restriction that $m^*_{0k} = m^*_{1k}$ for all $k= 1, \dots, K$.
This leaves the number of equations unchanged at $2K$, but reduces the number of unknowns to $K+3$.
The smallest $K$ for which $K+3$ is at least as large as $2K$ is 3, which makes it clear why \citeauthor{Lewbel}'s proof requires that the ``instrument'' take on at least three values.
%Footnote explaining why Lewbel can't use a regressor exog condition to get identification with a two-valued instrument: it's because he's leaving out a relevant regressor, namely z. This means that the error term of changes. See our whiteboard photo from 2015-09-18 14.04.59

Unlike \cite{Lewbel}, we, along with \cite{Mahajan} and others, assume that $z$ satisfies the exclusion restriction. 
This implies a different constraint on the $m^*_{tk}$ from Table \ref{tab:unobservables}.
Since $u = c + \varepsilon$, $\mathbb{E}[\varepsilon|z]=0$ implies that
\begin{equation}
  \mathbb{E}[u|z] = E[u] = c.
  \label{eq:validInstrument}
\end{equation}
By the Law of Iterated Expectations, this can be expressed as 
\begin{equation*}
  (1 - p_k^*)m_{0k}^* + p_k^* m_{1k}^* = c 
\end{equation*}
for all $k = 1, \dots, K$.
This restriction imposes that a particular weighted sum over the rows of a given column of Table \ref{tab:unobservables} takes the same value \emph{across} columns.
Using Equation \ref{eq:pkstar} and rearranging gives 
\begin{equation*}
  \frac{(1 - p_k - \alpha_1) m_{0k}^*}{1 - \alpha_0 - \alpha_1} = c - \frac{(p_k - \alpha_0)m_{1k^*}}{1 - \alpha_0 - \alpha_1}
\end{equation*}
which we can substitute into Equations \ref{eq:MC0} and \ref{eq:MC1} to yield
\begin{align}
  \label{eq:MC0IV}
  \hat{y}_{0k} &=\alpha_1(p_k - \alpha_0)\left(\frac{\beta}{1 - \alpha_0 - \alpha_1}\right) + (1-\alpha_0)c - (p _k -  \alpha_0)m_{1k}^* \\[1.5ex]
  \label{eq:MC1IV}
  \hat{y}_{1k} &=(1-\alpha_1)(p_k - \alpha_0)\left(\frac{\beta}{1 - \alpha_0 - \alpha_1}\right) + \alpha_0 c + (p _k -  \alpha_0)m_{1k}^*.
\end{align}
Equations \ref{eq:MC0IV} and \ref{eq:MC1IV} also make it clear why the IV estimator is inconsistent in the face of non-differential measurement error, and that this inconsistency does not depend on the endogeneity of the treatment, as noted by \cite{FL}.
Adding together Equations \ref{eq:MC0IV} and \ref{eq:MC1IV} yields
\begin{equation*}
  \hat{y}_{0k} + \hat{y}_{1k} = c + (p_k - \alpha_0)\left( \frac{\beta}{1 - \alpha_0 - \alpha_1} \right) 
\end{equation*}
completely eliminating the $m^*_{1k}$ from the system.
Taking the difference of the preceding expression expression evaluated at two different values of the instrument, $z_{k}$ and $z_{\ell}$, and rearranging
\begin{equation}
  \mathcal{W} = \frac{(\hat{y}_{0k} + \hat{y}_{1k}) - (\hat{y}_{0\ell} + \hat{y}_{1\ell})}{p_k - p_\ell} =  \frac{\beta}{1 - \alpha_0 - \alpha_1}
  \label{eq:wald}
\end{equation}
which is the well-known Wald IV estimator, since $\hat{y}_{0k} + \hat{y}_{1k} = \mathbb{E}[y|z = z_k]$.

Imposing Equation \ref{eq:validInstrument} replaces the $K$ unknown parameters $\left\{ m^*_{0k}\right\}_{k=1}^K $ with a single parameter $c$, leaving us with the same $2K$ equations but only $K+4$ unknowns.
When $K=2$ we have 4 equations and 6 unknowns.
So how can one identify $\beta$ in this case?
The literature has imposed additional assumptions which, using our notation, can once again be mapped into restrictions on the $m_{tk}^*$.
\cite{BBS}, \cite{KRS}, and \cite{Mahajan} make a \emph{joint} exogeneity assumption on $(T^*,z)$, namely $\mathbb{E}[\varepsilon|T^*,z]=0$.
Notice that this is strictly stronger than assuming that the instrument is valid and the treatment is exogenous.
In our notation, this joint exogeneity assumption is equivalent to imposing $m_{tk}^*=c$ for all $t,k$.
This reduces the parameter count to 4 regardless of the value of $K$.
Thus, then the instrument is binary, we have exactly as many equations as unknowns.
The arguments in \cite{BBS}, \cite{KRS}, and \cite{Mahajan} are all equivalent to solving Equations \ref{eq:MC0IV} and \ref{eq:MC1IV} for $\beta$ under the added restriction that $m^*_{1k}=c$, establishing identification for this case.
\cite{FL} use the same argument in a linear model with a potentially continuous instrument, but impose only the weaker conditions that the treatment is exogenous and the instrument is valid. 
Nevertheless, a crucial step in their derivation implicitly assumes the stronger joint exogeneity assumption used by \cite{BBS}, \cite{KRS} and \cite{Mahajan}.
Without this assumption, their proof does not in fact go through.
%Add a footnote showing the gap between the two sets of assumptions and how it relates to the continuous case, triple moment, etc.

If one wishes to allow for an endogenous treatment, clearly the joint exogeneity assumption $m_{tk}^*=c$ is unusable: we are back to $2K$ equations in $K+4$ unknowns.
Based on the identification arguments described above, there would seem to be two possible avenues for identification of the treatment effect when a valid instrument is available.
A first possibility would be to impose alternative conditions on the $m^*_{tk}$ that are compatible with an endogenous treatment.
If $z$ is binary, two additional restrictions would suffice to equate the counts of moments and unknowns.
This is the route followed by \cite{Mahajan} in his proof of identification with a binary instrument and endogenous treatment.
His Equation (11), expressed in our notation, amounts to adding the two restriction $m^*_{11} = m^*_{12}$ and $m^*_{01} = m^*_{02}$. 
Another possibility, suggested by \citeauthor{Lewbel}'s approach, would be to rely on an instrument that takes on more than two values.
Following this approach would suggest a 4-valued instrument, the smallest value of $K$ for which $2K = K+4$.
In the following section we present two of our main results: first \citeauthor{Mahajan}'s approach leads to a contradiction, and second, regardless of the value of $K$, $\beta$ is unidentified.  



%\cite{FL} point out that an instrument can be used in place of a second measure of $T^*$ provided that $T^*$ is still exogenous.
%Essentially the same estimator as in \cite{BBS} and \cite{KRS} but more general since the instrument need not be binary: can in fact be continuous.
%But they make a mistake. 
%They assume $E[zu]=0$, $E[T^*u]=0$ and non-differential measurement error and claim that this is sufficient to consistently estimate $\beta$.
%However, this is incorrect: we need the additional assumption that $E[zT^*u]=0$ which is stronger.
%(They seem to think that this term only appears when you have an endogenous $T^*$.)
%While \cite{FL} are aware that there are some differences between two measures of $T^*$, as in \cite{BBS}, and an arbitrary instrument $z$, they seem to have missed one subtle point.
%The assumptions in \cite{BBS} in fact imply that $E[u|T^*,z]=0$.\footnote{This follows from Assumptions A1 and A2 combined with Equation 3.}
%From this it follows that $E[zT^*u]=E[zu]=E[T^*u]=0$. 
%However, if one takes the non-differential measurement assumption literally it is in fact sufficient in the case of two measures to assume only that $E[zu]=E[T^*u]=0$:
%\begin{align}
%  E[zT^*u] &= E[(T^*+w)T^*u] = E[(T^*)^2u] + E[wuT^*]  \\
%  &= E\left[ E\left( u|T^* \right)(T^*)^2 \right] + E\left[E\left( wu|T^* \right)T^*\right]\\
%  &= 0 + E\left[ E(w|T^*)E(u|T^*)T^* \right] = 0
%\end{align}
%using the fact that $E[u|T^*]=0$ and $w$ is independent of $u$ conditional on $T^*$.
%This argument does \emph{not} necessarily apply to an arbitrary instrument $z$: $E[zu]=E[T^*u]=0$ does not imply that $E[zT^*u]=0$.
%\todo[inline]{Put in our simple binary example.}
%While it might seem strange to assume in practice that $E[zu]=E[T^*u]=0$ are exogenous but not that $E[zT^*u]=0$ the point is merely that this is an additional assumption beyond the usual assumptions of lack of correlation. 
%
%
%Mahajan uses the assumption $E[u|T^*,z]$ to get identification using same basic estimator as \cite{BBS}.
%
%Another closely related paper is \cite{Lewbel}.
%Whereas Mahajan makes sufficient assumptions to identify $\beta$ with a two-valued $z$, provided that $T^*$ is exogenous, Lewbel works with a three-valued $z$.
%While Lewbel also assumes that $E[T^*u]=0$, his ``instrument'' is really more like a covariate.
%He assumes that $z$ is unrelated to the mis-classification probabilities but allows it to have a direct effect on $y$, as long as there is no interaction between $T^*$ and $z$.
%Since this involves imposing fewer restrictions on the $m_{ij}$, Lewbel requires that $z$ take on more values.
%There is also some kind of determinant condition that we don't fully understand yet, but will figure out soon!
%
%
%
