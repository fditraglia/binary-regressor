%!TEX root = ./main.tex
\section{Main Results}
\subsection{Notation and Basic Properties of the Model}
Consider the model described in Equations \ref{eq:model}--\ref{eq:nondiff}.
Our arguments below, like those of \cite{Mahajan} and \cite{Lewbel}, proceed by holding the exogenous covariates \emph{fixed} at some level $\mathbf{x}_a$.
As such, there is no loss of generality from suppressing dependence on $\mathbf{x}$ in our notation. 
It should be understood throughout that any conditioning statements are evaluated at $\mathbf{x}=\mathbf{x}_a$.
To this end let $c = h(0,\mathbf{x}_a)$ and define $\beta = h(1,\mathbf{x}_a) - h(0,\mathbf{x}_a)$.
Using this notation, Equation \ref{eq:model} can be re-expressed as a simple linear model, namely 
\begin{equation}
  y = \beta T^* + u 
  \label{eq:linear}
\end{equation}
where we define $u = c + \varepsilon$, an error term that need not be mean zero.
We maintain throughout that $\beta \neq 0$.
If it were zero, this would imply that $T^*$ is irrelevant for $y$ which can be directly tested regardless of whether any mis-classification is present and regardless of whether $T^*$ is endogenous.\footnote{This is because, as we will see below, the Wald Estimator is identified and is proportional to the treatment effect. This estimator exists provided that we have a valid and relevant instrument that takes on at least two values.} 

From the perspective of non-parametric identification, the observables in this problem are the conditional distribution of $y$ given $(T,z)$, the conditional probabilities of $T$ given $z$, and the marginal probabilities of $z$.
For now, following the existing literature, we will restrict attention to the conditional mean of $y$.
Below we consider using higher moments of $y$.
Let $\bar{y}_{t,k}$ denote $\mathbb{E}[y|T=t,z=z_k]$, let $p_k$ denote $\mathbb{P}(T=1|z=z_k)$ and let $q_k = \mathbb{P}(z=z_k)$.
Table \ref{tab:observables} depicts the observable first moments for this problem.

\begin{table}
  \centering
  \begin{tabular}{c|c|c|c|c|}
    \multicolumn{1}{c}{}& \multicolumn{1}{c}{$z=1$} &\multicolumn{1}{c}{$z=1$} & \multicolumn{1}{c}{\dots} &\multicolumn{1}{c}{$z=K$}\\
    \cline{2-5}
    $T=0$ & \diagbox[dir=NE]{$\bar{y}_{01}$}{$p_{01}$} & \diagbox[dir=NE]{$\bar{y}_{02}$}{$p_{02}$} & \dots &\diagbox[dir=NE]{$\bar{y}_{0K}$}{$p_{0K}$}\\
    \cline{2-5}
    $T=1$ & \diagbox[dir=NE]{$\bar{y}_{11}$}{$p_{11}$} & \diagbox[dir=NE]{$\bar{y}_{12}$}{$p_{12}$} & \dots &\diagbox[dir=NE]{$\bar{y}_{1K}$}{$p_{1K}$}\\
    \cline{2-5}
  \end{tabular}
  \caption{Observables, using the shorthand $p_{0k}=q_k(1-p_k)$ and $p_{1k}=q_kp_k$.}
  \label{tab:observables}
\end{table}


The observed cell means $\bar{y}_{tk}$ depend on a number of unobservable parameters which we now define. 
Let $m^*_{tk}$ denote the conditional mean of $u$ given $T^*=t$ and $z=z_k$, $\mathbb{E}[u|T^*=t, z=z_k]$, and let $p^*_k$ denote $\mathbb{P}(T^*=1|z=z_k)$.
These quantities are depicted in Table \ref{tab:unobservables}.
By the Law of Total Probability and the definitions of $p_k$ and $p_k^*$,
\begin{eqnarray*}
  p_k &=&  \mathbb{P}(T=1|z=z_k,T^*=0)(1-p^*_{k}) + \mathbb{P}(T=1|z=z_k,T^*=1)p^*_k \\
  &=& \alpha_0 (1-p^*_k) + (1 - \alpha_1) p_k^* 
\end{eqnarray*}
since the misclassification probabilities do not depend on $z$ by Equations \ref{eq:a0}--\ref{eq:a1}.
Rearranging, 
\begin{equation}
  p_k^* = \frac{p_k - \alpha_0}{1 - \alpha_0 - \alpha_1}, \quad
  1 -p_k^* = \frac{1 - p_k - \alpha_1}{1 - \alpha_0 - \alpha_1}. 
  \label{eq:pkstar}
\end{equation}
Equation \ref{eq:pkstar} implies that $p_k^*$ is observable given knowledge of $\alpha_0$ and $\alpha_1$, since $p_k$ is observable.
Note that for these equations to be meaningful, we require that $\alpha_0 + \alpha_1 \neq 0$.
Indeed, the existing literature \citep{Mahajan,Lewbel,FL,BBS,KRS} imposes the stronger condition that $\alpha_0 + \alpha_1 < 1$, namely that the measurement error is not so severe that $1-T$ is a better predictor of $T^*$ than $T$ is, and vice-versa.
In its absence the treatment effect would only be identified up to sign.
Our identification and partial identification results, presented below, will not in fact require that $\alpha_0 + \alpha_1 < 1$.

\begin{table}
  \centering
  \begin{tabular}{c|c|c|c|c|}
    \multicolumn{1}{c}{}& \multicolumn{1}{c}{$z=1$} &\multicolumn{1}{c}{$z=1$} & \multicolumn{1}{c}{\dots} &\multicolumn{1}{c}{$z=K$}\\
    \cline{2-5}
    $T^*=0$ & \diagbox[dir=NE]{$m^*_{01}$}{$p^*_{01}$} & \diagbox[dir=NE]{$m^*_{02}$}{$p^*_{02}$} & \dots &\diagbox[dir=NE]{$m^*_{0K}$}{$p^*_{0K}$}\\
    \cline{2-5}
    $T^*=1$ & \diagbox[dir=NE]{$m^*_{11}$}{$p^*_{11}$} & \diagbox[dir=NE]{$m^*_{12}$}{$p^*_{12}$} & \dots &\diagbox[dir=NE]{$m^*_{1K}$}{$p^*_{1K}$}\\
    \cline{2-5}
  \end{tabular}
  \caption{Unobservables, using the shorthand $p^*_{0k}=q_k(1-p^*_k)$ and $p^*_{1k}=q_kp_k^*$.}
  \label{tab:unobservables}
\end{table}

A key assumption below will be the conditional mean independence of the error term and instrument, in other words $\mathbb{E}\left[ \varepsilon|z \right]=0$.
Since we have defined $u = c + \varepsilon$, this assumption can be expressed in terms of $m^*_{tk}$ as
\begin{equation}
  (1 - p_k^*)m_{0k}^* + p_k^* m_{1k}^* = c 
\end{equation}
for all $k = 1, \dots, K$.
This restriction imposes that a particular weighted sum over the rows of a given column of Table \ref{tab:unobservables} takes the same value \emph{across} columns.

