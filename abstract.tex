%!TEX root = ./main.tex
This paper studies the use of a discrete instrumental variable to identify the causal effect of a endogenous, mis-measured, binary treatment in a homogeneous effects model with additively separable errors.
We begin by showing that the only existing identification result for this case, which appears in \cite{Mahajan}, is incorrect.
As such, identification in this model remains an open question.
We provide a convenient notational framework to address this question and use it to derive a number of results.
First, we prove that the treatment effect is unidentified based on conditional first-moment information, regardless of the number of values that the instrument may take.
Second, we derive a novel partial identification result based on conditional second moments that can be used to test for the presence of mis-classification and to construct bounds for the treatment effect.
In certain special cases, we can in fact obtain point identification of  treatment effect based on second moment information alone.
When this is not possible, we show that adding conditional third moment information point identifies the treatment effect and completely characterizes the measurement error process.
