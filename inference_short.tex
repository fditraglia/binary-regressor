%!TEX root = ./main.tex
\section{GMM Inference}
\label{sec:inference}

Lemmas \ref{lem:wald}--\ref{lem:eta3} yield a system of linear moment equations in $\boldsymbol{\theta}(\mathbf{x}) = \big(\theta_1(\mathbf{x}), \theta_2(\mathbf{x}),\theta_3(\mathbf{x})\big)$.
Defining a vector of intercepts $\boldsymbol{\kappa}(\mathbf{x}) = \big(\kappa_1(\mathbf{x}), \kappa_2(\mathbf{x}), \kappa_3(\mathbf{x})\big)$,
and a vector of observables $\mathbf{w}' = (T, y, yT, y^2, y^2 T, y^3)$ we can write this system as
\begin{equation}
\mathbb{E}\left[
  \bigg\{\boldsymbol{\Psi}\big(\boldsymbol{\theta}(\mathbf{x})\big)\mathbf{w}_i - \boldsymbol{\kappa}(\mathbf{x})\bigg\} \otimes 
\left(
\begin{array}{c}
  1 \\ z
\end{array}\right)\Bigg| \mathbf{x} = \boldsymbol{x}
\right] = \mathbf{0}
\label{eq:MCs_endog}
\end{equation}
where we define the matrix of functions $\boldsymbol{\Psi}\big(\boldsymbol{\theta}(\mathbf{x})\big)$ as follows:
\[
  \boldsymbol{\Psi}\big(\boldsymbol{\theta}(\mathbf{x})\big) = 
  \left[
  \begin{array}{rrrrrr}
    -\theta_1(\mathbf{x}) & 1 & 0 & 0 & 0 & 0\\
    \theta_2(\mathbf{x}) & 0 & -2\theta_1(\mathbf{x}) & 1 & 0 & 0\\ 
    -\theta_3(\mathbf{x}) & 0 & 3\theta_2(\mathbf{x}) & 0 & -3\theta_1(\mathbf{x}) & 1
\end{array}\right].
\]
Via the mapping from reduced form to structural parameters from  Equations \ref{eq:theta1_def}--\ref{eq:theta3_def}

$\boldsymbol{\theta}(\mathbf{x})$
where, suppressing dependence on $\mathbf{x}$ for simplicity, $\boldsymbol{\Psi} = \left[
  \begin{array}{ccc}
    \boldsymbol{\psi}_1 & \boldsymbol{\psi_2} & \boldsymbol{\psi_3}
\end{array}\right]$ and
\begin{equation}
  \boldsymbol{\psi}_1' = (-\theta_1, 1, 0, 0, 0, 0), \quad
  \boldsymbol{\psi}_2' = (\theta_2, 0, -2\theta_1, 1, 0, 0), \quad
  \boldsymbol{\psi}_3' = (-\theta_3, 0, 3\theta_2, 0, -3\theta_1, 1)
  \label{eq:psi_def}
\end{equation}



Non-linearity arises solely through the relationship between the reduced from parameters $\boldsymbol{\theta}$ and the structural parameters $(\alpha_0, \alpha_1, \beta)$.
To convert the preceding moment equations into unconditional moment equalities, we define the additional reduced form parameters $\boldsymbol{\kappa} = (\kappa_1, \kappa_2, \kappa_3)$ and observables 

\begin{align*}
\kappa_1 &= c - \alpha_0 \theta_1\\
  \kappa_2 &= c^2 + \sigma_{\varepsilon\varepsilon} + \alpha_0 (\theta_2 - 2c \theta_1)\\
  \kappa_3 &= c^3 + 3\left( c - \theta_1 \alpha_0 \right) \sigma_{\varepsilon\varepsilon} + \mathbb{E}[\varepsilon^3] - \alpha_0 \theta_3 - 3 c \alpha_0 \left[ \theta_1 \left( c + \beta \right) - 2\theta_1^2 (1 - \alpha_1) \right]
\end{align*}
Building on this notation, let
\begin{equation}
  \boldsymbol{\psi}_1' = (-\theta_1, 1, 0, 0, 0, 0), \quad
  \boldsymbol{\psi}_2' = (\theta_2, 0, -2\theta_1, 1, 0, 0), \quad
  \boldsymbol{\psi}_3' = (-\theta_3, 0, 3\theta_2, 0, -3\theta_1, 1)
\end{equation}
and collect these in the matrix
$\boldsymbol{\Psi} = \left[
  \begin{array}{ccc}
    \boldsymbol{\psi}_1 & \boldsymbol{\psi_2} & \boldsymbol{\psi_3}
\end{array}\right]$.
Defining the observed data vector $\mathbf{w}_i' = (T_i, y_i, y_iT_i, y_i^2, y_i^2 T_i, y_i^3)$ for observation $i$, we can re-write the moment equations as:
\begin{equation}
\mathbb{E}\left[
  \big(\boldsymbol{\Psi}'(\boldsymbol{\theta})\mathbf{w}_i - \boldsymbol{\kappa}\big) \otimes 
\left(
\begin{array}{c}
  1 \\ z_i
\end{array}\right)
\right] = \mathbf{0}.
\end{equation}

Equation \ref{eq:MCs_endog} is a just-identified, linear system of moment equalities in the reduced form parameters $(\boldsymbol{\theta},\boldsymbol{\kappa})$ and yields explicit GMM estimators $(\widehat{\boldsymbol{\kappa}},\widehat{\boldsymbol{\theta}})$.
From Theorem \ref{thm:main_ident}, knowledge of $\boldsymbol{\theta}$ suffices to identify $\beta$.

\begin{equation}
\mathbb{E}\left[
  \bigg\{\boldsymbol{\Psi}\big(\alpha_0(\mathbf{x}), \alpha_1(\mathbf{x}), \beta(\mathbf{x})\big)' \mathbf{w}_i - \boldsymbol{\kappa}(\mathbf{x})\bigg\} \otimes 
\left(
\begin{array}{c}
  1 \\ z_i
\end{array}\right)\Bigg| \mathbf{x}_i = \boldsymbol{x}
\right] = \mathbf{0}.
\end{equation}

\todo[inline]{Just a short section sketching how to carry out inference. Show the GMM conditions, talk about how to use them. Perhaps cite Lewbel (2005) so that we can handle covariates. Mention that there is a weak identification problem when $\beta$ is small. Mention that one possible way to address this is to use tools from the moment inequality literature. Refer to our NBER working paper for more details.}

\cite{DiTragliaGarciaWP2017} for more details

\cite{Lewbel2007} proposes blah
